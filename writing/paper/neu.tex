% !TEX TS-program = lualatexmk
% glossa-template.tex
% Copyright 2016 Guido Vanden Wyngaerd
%
% This work may be distributed and/or modified under the
% conditions of the LaTeX Project Public License.
% The latest version of this license is in
%   http://www.latex-project.org/lppl.txt
% and version 1.3 or later is part of all distributions of LaTeX
% version 2005/12/01 or later.
%
% This work has the LPPL maintenance status `maintained'.
% 
% The Current Maintainer of this work is 
% Guido Vanden Wyngaerd (guido.vandenwyngaerd@kuleuven.be).
%
% This work consists of the files 
% glossa.cls
% glossa.bst
% gl-authoryear-comp.cbx
% biblatex-gl.bbx
% glossa-template.tex
% glossa.png
%
% The files of the work are derived from the Semantics & Pragmatics style files
% by Kai von Fintel, Christopher Potts, and Chung-chieh Shan
% All changes are documented on the github repository 
% https://github.com/guidovw/Glossalatex.

\PassOptionsToPackage{table}{xcolor}
\documentclass[times,linguex,xcolor]{glossa}
\usepackage{rotating}
\usepackage{tablefootnote}
\usepackage{colortbl}
\usepackage{color}
\usepackage{multicol}
\usepackage{booktabs}

\usepackage{adjustbox}
\usepackage{array}

\newcolumntype{R}[2]{%
    >{\adjustbox{angle=#1,lap=\width-(#2)}\bgroup}%
    l%
    <{\egroup}%
}

\newcommand*\rots{\multicolumn{1}{R{90}{.7em}}}% no optional argument here, please!
%\usepackage{xcolor}

% possible options:
% [times] for Times font (default if no option is chosen)
% [cm] for Computer Modern font
% [lucida] for Lucida font (not freely available)
% [brill] open type font, freely downloadable for non-commercial use from http://www.brill.com/about/brill-fonts; requires xetex
% [charis] for CharisSIL font, freely downloadable from http://software.sil.org/charis/
% for the Brill an CharisSIL fonts, you have to use the XeLatex typesetting engine (not pdfLatex)
% for headings, tables, captions, etc., Fira Sans is used: https://www.fontsquirrel.com/fonts/fira-sans
% [biblatex] for using biblatex (the default is natbib, do not load the natbib package in this file, it is loaded automatically via the document class glossa.cls)
% [linguex] loads the linguex example package
% !! a note on the use of linguex: in glossed examples, the third line of the example (the translation) needs to be prefixed with \glt. This is to allow a first line with the name of the language and the source of the example. See example (2) in the text for an illustration.
% !! a note on the use of bibtex: for PhD dissertations to typeset correctly in the references list, the Address field needs to contain the city (for US cities in the format "Santa Cruz, CA")

%\addbibresource{sample.bib}
% the above line is for use with biblatex
% replace this by the name of your bib-file (extension .bib is required)
% comment out if you use natbib/bibtex

\let\B\relax %to resolve a conflict in the definition of these commands between xyling and xunicode (the latter called by fontspec, called by charis)
\let\T\relax
\usepackage{xyling} %for trees; the use of xyling with the CharisSIL font produces poor results in the branches. This problem does not arise with the packages qtree or forest.
%\usepackage[linguistics]{forest} %for nice trees!


% \pdf* commands provide metadata for the PDF output. ASCII characters only!
\pdfauthor{}
\pdftitle{What is at-issueness?}
\pdfkeywords{}

\title[What is at-issueness?]{What is at-issueness? An experimental comparison of diagnostics\\ 
  % \bigskip \large Word count: 4720
  }
% Optional short title inside square brackets, for the running headers.

\author[]% short form of the author names for the running header. If no short author is given, no authors print in the headers.
{%as many authors as you like, each separated by \AND.
  % \spauthor{Waltraud Paul\\
  % \institute{CRLAO, CNRS-EHESS-INALCO}\\
  % \small{%105, Bd. Raspail, 75005 Paris\\
  % waltraud.paul@ehess.fr}
  % }
  % \AND
  % \spauthor{Guido Vanden Wyngaerd \\
  % \institute{KU Leuven}\\
  % \small{%Warmoesberg 26, 1000 Brussel\\
  % guido.vandenwyngaerd@kuleuven.be}
  % }%
}


%=====================================================================
%=========================== text ===========================

	% punctuation
		\usepackage{csquotes} % for quotation marks

%====================================================================
%=========================== links, references =======================
	% more linguex options for referencing select examples without parentheses
	  \newif\ifparens\parensfalse
	  \makeatletter
	  \renewcommand{\theExNo}{\protect\theExLBr\arabic{ExNo}\protect\theExRBr}
	  \renewcommand{\theSubExNo}{%
	    \hbox{\if@noftnote\protect\theExLBr\Exarabic{ExNo}\firstrefdash
	        \Exalph{SubExNo}\protect\theExRBr
	      \else
	        \protect\theFnExLBr\Exroman{FnExNo}\firstrefdash%
	        \Exalph{SubExNo}\protect\theFnExRBr
	      \fi}}

	  \renewcommand{\theSubSubExNo}{%
	    \hbox{\if@noftnote\protect\theExLBr%
	            \Exarabic{ExNo}\firstrefdash\Exalph{SubExNo}\secondrefdash
	               \Exroman{SubSubExNo}\protect\theExRBr%
	      \else\protect\theFnExLBr\Exroman{FnExNo}\firstrefdash
	                \Exalph{SubExNo}\secondrefdash\Exarabic{SubSubExNo}\protect\theFnExRBr\fi}}%
	  \makeatother
	  \renewcommand\theExLBr{\ifparens\else(\fi}
	  \renewcommand\theExRBr{\ifparens\else)\fi}
	  \newcommand\pref[1]{{\parenstrue\ref{#1}}}

	% in text citation macros
	\newcommand{\citepos}[1]{\citeauthor{#1}'s \citeyear{#1}}
	\newcommand{\citeposs}[1]{\citeauthor{#1}'s}
	\newcommand{\citetpos}[1]{\citeauthor{#1}'s \citeyear{#1}}

	%
	\usepackage{cleveref}



%=====================================================================
%=========================== figures, tables =========================

	\usepackage{enumitem}


%=====================================================================
%=========================== figures, tables =========================
	% \usepackage{subcaption}
	\usepackage{multirow}

% positive coefficients/difference
\definecolor{purple1}{RGB}{178,24,43}
\definecolor{purple2}{RGB}{239,138,98} 
\definecolor{purple3}{RGB}{253,219,199} 
%\definecolor{yellow4}{RGB}{255,255,204}
%\definecolor{green1}{RGB}{0, 158, 115} % >.95
%\definecolor{green2}{RGB}{55, 185, 141} % .85-.95
%\definecolor{green3}{RGB}{88, 214, 167} % .75-.85
%\definecolor{green4}{RGB}{119, 242, 194} % <.75

% negative coefficients/difference
\definecolor{yellow1}{RGB}{33,102,172}
\definecolor{yellow2}{RGB}{103,169,207}
\definecolor{yellow3}{RGB}{209,229,240}
%\definecolor{purple4}{RGB}{236,206,223}

\begin{document}


\maketitle


\begin{abstract}
  At-issueness is a key concept in theoretical semantics/pragmatics, but there is no consensus about how it is defined or diagnosed (e.g., \citealt{tonhauser_diagnosing_2012,tonhauser_how_2018,koev_notions_2018}). We present experimental data investigating whether four widely used diagnostics for at-issueness yield consistent results. Our findings reveal significant differences across diagnostics, indicating they are not interchangeable. Since the diagnostics target distinct theoretical conceptions of at-issueness, these differences offer insight into their comparability.

\end{abstract}

% \begin{keywords}
%   at-issueness, experimental pragmatics, discourse interpretation
% \end{keywords}

\section{Introduction \label{sec:1_introduction}}

  At-issueness is a key concept in theoretical semantics and pragmatics, distinguishing between at-issue propositions conveyed by an utterance, those contributing to its main point, and those that do not (e.g., \citealt{karttunen_conventional_1979,horton_presuppositions_1988,abbott_presuppositions_2000,faller_semantics_2003,potts_logic_2005,tonhauser_diagnosing_2012}). Despite its importance, the concept lacks a unified definition. Instead, various theoretical notions (\citealt{koev_notions_2018,tonhauser_how_2018}) and empirical diagnostics (e.g., \citealt{tonhauser_diagnosing_2012}) have been proposed. This paper addresses the question whether four widely‐used diagnostics for at-issueness yield consistent results when testing the same stimuli. Our findings reveal significant differences across diagnostics, indicating they are not interchangeable. Since the diagnostics target distinct theoretical conceptions of at-issueness, these differences offer insight into the comparability of these conceptions.

  The four diagnostics we tested are illustrated in (\pref{qud}--\pref{yesbut}) for sentence-medial non-restrictive relative clauses (NRRCs). As these are usually taken to contribute non-at-issue content, participants are expected to: Give low naturalness ratings under the QUD diagnostic \ref{qud} and the direct dissent diagnostic \ref{dd}, choose one of the \emph{yes}-responses under the `yes, but' diagnostic in \ref{yesbut}, and not interpret the speaker to be asking about the content under the `asking-whether' diagnostic in \ref{aw}.

  \ex. \label{qud}%
    QUD diagnostic (e.g., \citealt{tonhauser_diagnosing_2012,chen_presuppositions_2024})
    \a.[A:] \emph{What did Greg buy?}
    \b.[B:] \emph{Greg, who bought a new car, is envied by his neighbor.}
    \z.
    Question to participants: How well does B's response fit A's question?
  \z.

  \ex. \label{dd} Direct dissent diagnostic (e.g., \citealt{tonhauser_diagnosing_2012,syrett_experimental_2015})
    \a.[A:] \emph{Greg, who bought a new car, is envied by his neighbor.}
    \b.[B:]\emph{No, that's not true, he didn't buy a new car.}
    \z.
  Question to participants: How natural is B's rejection of A's utterance?
  \z.

  \ex. \label{yesbut}%
    `yes, but' diagnostic (e.g., \citealt{xue_correlation_2011,destruel_cross-linguistic_2015})
    \a.[A:] \emph{Greg, who bought a new car, is envied by his neighbor.}
    \b.[B:] \emph{Yes, but he didn't buy a new car.} /
    \b.[] \emph{Yes, and he didn't buy a new car.} /
    \b.[] \emph{No, he didn't buy a new car.}
    \z.
    Task for participants: Choose the response that sounds best.
  \z.

  \ex. \label{aw}%
    `asking whether' diagnostic (e.g., \citealt{tonhauser_how_2018,solstad_cataphoric_2024})\smallskip\\
      \emph{Is Greg, who bought a new car, envied by his neighbor?}\smallskip
  \\ Question to participants: Is the speaker asking whether Greg bought a new car?
  \z.

  \citealt{koev_notions_2018} suggests that different diagnostics reflect distinct theoretical conceptions of at-issueness: First, the QUD diagnostic \ref{qud} aligns with Q(uestion)-at-issueness (\citealt{simons_what_2010}), which conceptualizes at-issue content as addressing a question under discussion (QUD; \citealt{roberts_information_1996,ginzburg_interrogatives_1996}) established in prior discourse (\citealt{amaral_review_2007}).
  %
  Second, the direct dissent \ref{dd} and `yes, but' diagnostics \ref{yesbut} reflect P(roposal)-at-issueness (\citealt{koev_apposition_2013}), under which the at-issue content of an utterance constitutes its main assertion, understood as a proposal to update the common ground. Such proposals can thus be directly affirmed or denied using default discourse moves like polar response particles (PRPs; e.g., English \emph{yes/no}; \citealt{farkas_reacting_2010}). In contrast, non-at-issue content is either presupposed (already entailed in the common ground; \citealt{stalnaker_presuppositions_1973,stalnaker_common_2002}), or newly imposed on the common ground (\citealt{murray_varieties_2014,anderbois_at-issue_2015}), and requires special moves for disagreement, like revision, correction, or negotiation (\citealt{potts_logic_2005}).
  %
  Finally, the `asking whether' diagnostic (\citealt{tonhauser_how_2018}) characterizes the at-issue content of questions as explicitly raises a QUD, whereas their non-at-issue content does not contribute to what the QUD is (following \citealt{roberts_information_1996}). While closely related to Q-at-issueness, this diagnostic does not fully align with Koev’s Q/P distinction, a point we revisit in the discussion (§\ref{sec:discussion}).

  \subsection{medial appositives}
  Prior research has led to diverging conclusions about the at-issueness of certain types of content, potentially arising from differences between the diagnostics: \citealt{koev_notions_2018} suggests that English medial appositive RCs can be Q-at-issue, but not P-at-issue, based on impressionistic judgment data: they behave as at-issue for the QUD diagnostic, but not for the direct dissent test. This is in line with \citepos{tonhauser_diagnosing_2012} findings for Paraguayan Guaraní medial appositive DPs.
  % (e.g., \emph{chesy angiru} \lq my mother's friend\rq).
  These behave as not-at-issue according to most diagnostics tested there (including direct dissent, and `yes, but'), but yielded mixed results with the QUD-diagnostic.
  \begin{itemize}
    \item experimental work in ... where propositions contributed by medial appositives are among the contents receiving the lowest at-issueness ratings under the ... diagnostics

    \begin{itemize}
      \item different results from the QUD-diagnostic \citealt{chen_presuppositions_2024}, Experiments 1, 2 found that attempting to address questions with the content of German medial appositive RCs lead to much lower question-answer match ratings ($\mu \approx 2$ out of a five-point scale) than main clauses ($\mu \approx 4.1$)\footnote{Notably, their Exp. 1 found that this effect is only present tendentially for 4--6 year old children, and Exp. 2 found that it is present but a bit weaker in adult second language learners of German (with first language Mandarin Chinese).}
      However, in this study, the NRRCs contained the expression \emph{übrigens} \lq by the way\rq, which Chen suggests is a marker of non-at-issue status of the clause.

      \item \citealt{destruel_cross-linguistic_2015}: when German medial appositives are contradicted in the following utterance, then most participants choose to signal agreement and contrast \emph{yes, but} ($\approx 90\%$), suggesting NAI status.

      \item \citealt{tonhauser_how_2018}, Medial appositives are among the contents that get the lowest ratings for asking-whether diagnostic in their Exp.1a, suggesting that these are NAI; 

      \item Solstad + Bott 2024
    \end{itemize}

  \end{itemize}

    

  \subsection{final appositives}
    \begin{itemize}
      \item these diagnostics have led to arguments that sentence-final appositives are more at-issue than their medial counterparts

      \item found that sentence-medial appositives are less at-issue than sentence-final ones using the direct dissent test.

      \item These two diagnostics are characterized by \citealt{koev_notions_2018} as \emph{forward-looking}, as they test whether a given content is at-issue relative to utterances in the following discourse.


      \item \citealt{syrett_experimental_2015} Their Exp. 2 found that given a choice to disagree with a preceding main clause or appositive content, participants choose disagreeing with the main clause over a medial appositive around 80\% of the time. However, for final medial RCs, this proportion is reduced to around 65\%. They conclude that final appositive clauses can compete with main clause content in the direct-dissent diagnostic, allowing these contents to be more readily interpreted as at-issue.

      \item see also corpus study in ander bois et al cited in S+K

      \item drozdov: no differences

    \end{itemize}
  
    In contrast, sentence-final appositives have been argued to be less at-issue than sentence-medial ones, that is, using 

    \begin{itemize}
      \item \citealt{koev_notions_2018}: English sentence-final slifting parentheticals (e.g., \emph{Ellen is a passionate cook, her fiancé claimed;} p. 11): these behave as at-issue based on the direct-dissent but not the QUD diagnostic.
    \end{itemize}


  Those findings, are, in turn, in line with the experimental study in, which 

  


  \begin{tabular}{l c c}\toprule
                                    & medial         & final        \\
                                    & appositives    & appositives    \\\midrule
      \citealt{koev_notions_2018}, QUD-diagnostic 
                                    & \multirow{2}{*}{AI}
                                                    & \multirow{2}{*}{--} \\ 
      \scriptsize English, impressionistic judgments  &  \\ \midrule

      \citealt{tonhauser_diagnosing_2012}, QUD     & \multirow{2}{*}{?}
                                                    & \multirow{2}{*}{--} \\ 
      \scriptsize P. Guaraní, fieldwork elicitation  &  \\ \midrule

      \citealt{chen_presuppositions_2024}, QUD     & \multirow{2}{*}{NAI}
                                                    & \multirow{2}{*}{--} \\ 
      \scriptsize German, 5-point rating experiment  &  \\ \midrule

      \citealt{destruel_cross-linguistic_2015}, `yes but'     & \multirow{2}{*}{NAI}
                                                    & \multirow{2}{*}{--} \\ 
      \scriptsize German, forced-choice continuation  &  \\ \midrule

      \citealt{koev_notions_2018}, direct dissent     & \multirow{2}{*}{NAI}
                                                    & \multirow{2}{*}{--} \\ 
      \scriptsize English, impressionistic judgments  &  \\ \midrule

      \citealt{tonhauser_diagnosing_2012}, direct dissent     & \multirow{2}{*}{NAI}
                                                    & \multirow{2}{*}{--} \\ 
      \scriptsize P. Guaraní, fieldwork elicitation  &  \\ \midrule

      \citealt{syrett_experimental_2015}, direct dissent     & \multirow{2}{*}{NAI}
                                                    & \multirow{2}{*}{AI} \\ 
      \scriptsize English, forced-choice continuation  &  \\ \midrule

  \end{tabular}

  \subsection{presupposed content}

    \begin{itemize}
      \item \citealt{chen_presuppositions_2024}, Exps. 1,2, QUD diagnostic difference between German \enquote{soft triggers} (\emph{entdecken,} ‘discover’; \emph{gewinnen,} ‘win’, \emph{schaffen,} ‘manage to’), whose content can be used to felicitously address a previous question (mean Q+A match rating: $\mu \approx 3.7$), whereas \enquote{hard triggers} (\emph{auch,} ‘too’; \emph{wieder,} ‘again’; it-clefts) came with significantly lower ratings ($\approx 2$). This effect was found for adult partipants, but only present tendentially on 4--6 year old children (Exp. 1), and present, but weaker in L2 learners of German with L1 Mandarin Chinese.

      \item \citealt{xue_correlation_2011}, Exp. 2, use the `yes, but' diagnostics to assess the (non-)at-issueness of the projective content of German expressions \emph{wissen} `know',  \emph{entdecken} `discover',  \emph{auch} `too',  \emph{wieder} `again',  

    \end{itemize}

    \begin{tabular}{l c c c c}\toprule
            & \emph{too} 
              & \emph{again}
                        \\\midrule
        
        \citealt{chen_presuppositions_2024}, QUD
            & \multirow{2}{*}{NAI}
              & \multirow{2}{*}{NAI}
                        \\ 
        \scriptsize German, 5-point rating experiment  &  \\ \midrule

        \citealt{xue_correlation_2011}, `yes, but'
            & \multirow{2}{*}{NAI}
              & \multirow{2}{*}{NAI}
                        \\ 
        \scriptsize German, forced-choice continuation  &  \\ \midrule

    \end{tabular}

  \subsection{clause-embedding predicates}
    \begin{itemize}
      \item \citealt{tonhauser_how_2018}
      \item degen + tonhauser
    \end{itemize}

    \begin{itemize}
      \item embedded content in sentences with clause-embedding predicates 
      \item the projection of this content is modulated by its at-issueness
      \item testing the at-issueness of 

    \end{itemize}

    \begin{tabular}{l c c c c}\toprule
            & \emph{discover}
              & \emph{know}
                & \emph{find out}
                        \\\midrule
        
        \citealt{chen_presuppositions_2024}, QUD
            & \multirow{2}{*}{AI}
              & \multirow{2}{*}{--}
                & \multirow{2}{*}{--}
                        \\ 
        \scriptsize German, 5-point rating experiment  &  \\ \midrule

        \citealt{xue_correlation_2011}, `yes, but'
            & \multirow{2}{*}{--}
              & \multirow{2}{*}{AI}
                & \multirow{2}{*}{?}
                        \\ 
        \scriptsize German, forced-choice continuation  &  \\ \midrule

    \end{tabular}

  \subsection{How we will compare the diagnostics}

\section{Assessing at-issueness}
  \subsection{QUD-diagnostic}
    The QUD diagnostic tests whether a propositional content associated with a declarative assertion (\pref{qud}B) can be interpreted as Q-at-issue by testing whether the assertion is felicitous as an answer to a preceding question that targets that content.
    \begin{itemize}
      \item show here
    \end{itemize}
    % The notion that the at-issueness of a content is related to whether it is used to address the QUD , defined explicitly in \citealt{simons_what_2010}, is labeled Q(uestion)-at-issueness in \citepos{koev_notions_2018} overview:

    \ex. \label{def:qai}%
      Q-at-issueness: \hfill (based on \citealt{simons_what_2010}: 26, \citealt{koev_notions_2018}: 2)\\
        A content $m$ is Q-at-issue in a context $c$ iff
        \a. \label{def:qai-relevant}%
          $m$ is relevant to the QUD in $c$, and
        \b.  \label{def:qai-conventional}%
          $p$ is appropriately conventionally marked relative to the QUD.
        \z. 
      \z.

      Here, $m$ may be either a propositional content or a question meaning. Relevance to the QUD is defined as follows:

      \ex. Relevance to the QUD in context $c$ \hfill (based on \citealt{simons_what_2010}: 13)
        \a. A proposition $p$ is relevant the QUD iff it contextually entails in $c$ a partial or complete answer to the QUD.
        \b. A question $q$ is relevant to the QUD, iff it has an answer that is relevant to the QUD.
        \z.
      \z.

      The QUD-diagnostic from \citealt{tonhauser_diagnosing_2012} operationalizes Q-at-issueness through naturalness judgments. It builds on two assumptions:
      \begin{enumerate}
        \item An overt question explicitly introduces a QUD.\footnote{add reference}
        \item An utterance is felicitous only if its at-issue content is relevant to the QUD (\citealt{amaral_review_2007,tonhauser_diagnosing_2012}).
      \end{enumerate}

      Koev suggests that this diagnostic is \emph{backward-looking}, because it tests whether a given content is at-issue relative to the previous discourse.

      \noindent To test whether a given content $m$ can be construed as Q-at-issue, participants are presented with a context that establishes a QUD via an overt question, followed by a response that includes $m$. For instance, \ref{qud} is used to diagnose the status of the content $m$ of the appositive RC (Greg bought a car) conveyed by B's utterance $U$, by presenting it as a response to a question $Q$ that $m$ is relevant to (What did Greg buy?), and asking a naturalness rating for $U$ as a response to $Q$.

      \ex.[\ref{qud}]
        \a.[A:] \emph{What did Greg buy?}
        \b.[B:] \emph{Greg, who bought a new car, is envied by his neighbor.}
        \z.
        Question to participants: How well does B's response fit A's question?
      \z.

      If $m$ (Greg bought a car) is interpreted as addressing the QUD, the response should receive high naturalness ratings. However, responses like (\pref{qud}B) typically receive low ratings, suggesting that $m$ is not at-issue, that is, even though $m$ is relevant to $Q$ and thereby satisfies the first part of the definition in \ref{def:qai-relevant}. The low naturalness should, therefore, reflect that $m$ is not-at-issue due to the second part of the definition in \ref{def:qai-conventional}: The low ratings for (\pref{qud}B) support the claim that appositive RCs are not appropriately conventionally marked to contribute at-issue content.

      \begin{itemize}
        \item \citealt{chen_presuppositions_2024} used this diagnostic 

        comparing  

      \end{itemize}

  \subsection{Direct dissent and `yes, but' diagnostic}
    Accordingly, the direct-dissent diagnostic \ref{dd} tests whether a proposition associated with the initial utterance (\pref{dd}A) can be interpreted as P-at-issue by testing whether it can be felicitously contradicted using \emph{no}. Relatedly, the `yes, but' diagnostic \ref{yesbut} assesses whether speakers prefer to signal agreement (using \emph{yes}) or disagreement (using \emph{no}) with the main assertion when contradicting the tested content.

    The direct dissent diagnostic \ref{dd} and the `yes, but' diagnostic \ref{yesbut} reflect the notion of P(roposal)-at-issueness, based on the assumption that at-issue content contributes to the main assertion of an utterance, which is taken to constitute a proposal to update the common ground.

     \ex. P-at-issueness: \hfill (\citealt{koev_apposition_2013,koev_notions_2018})\\
      A proposition $p$ is P-at-issue in a context $c$ iff
      \a. $p$ is a proposal in $c$ and
      \b. $p$ has not been accepted or rejected in $c$.
      \z.
    \z.

    Under this conception, the at-issue assertion is the contribution of an utterance that can be directly assented or dissented with using default discourse moves (in the sense of \citealt{farkas_reacting_2010}), for instance, using polar response particles (like English \emph{yes/no}). Conversely, non-at-issue content is assumed to be entailed by the common ground prior to the utterance in question (e.g., presupposed content, \citealt{stalnaker_presuppositions_1973,stalnaker_common_2002}), or imposed on the common ground (\citealt{murray_varieties_2014,anderbois_at-issue_2015}). Importantly, the diagnostics in \ref{dd} and \ref{yesbut} build on the assumption that non-at-issue content requires non-default discourse moves (such as revision, correction, or negotiation) to be dissented with.

    \begin{itemize}
      \item that it should be possible to signal (at least partial) agreement with the main assertion of an utterance even when contradicting is non-at-issue content. It tests whether speakers prefer signaling agreement or disagreement with the previous assertion (using \emph{yes/no}) in repsonses that contradict the content in question.

    \end{itemize}

    \begin{itemize}
      \item \citealt{syrett_experimental_2015}, Exp. 2: used a variant of the direct-dissent diagnostic within a forced-choice continuation task. 
      \item utterance that included some appositive content (illustated here for a medial appositive RC)

      polarity particle \emph{no} to disagree  choice to disagree with the main clause content, or the apposistive content.
    \end{itemize}

    \ex. \a.[A] My friend Sophie, \underline{who performed a piece by Mozart,} is a classical violinist.
      \b.[B1:] No, she’s not. (target: main clause)
      \b.[B2:] No, she didn’t. (target: appositive)
      \z. \z. 

      to avoid concerns that the choice about which proposition to disagree with may be affected by the participants opinion about the content of these propositions, we instead chose a version of the direct dissent task that more directky targets the question whether disagrement using \emph{no} is acceptable, by using acceptabiltiy judgments.

  \subsection{Asking whether}
    This diagnostic tests whether a content associated with a polar question \ref{aw} can be interpreted as at-issue by testing whether informands will understand it as the main issue being asked about.

    Because the definition in \ref{def:qai} references the preceding context, \citet{koev_notions_2018} suggests that QUD-at-issueness is a backward-looking notion of at-issueness. However, overt questions may explicitly raise a QUD\footnote{add reference}, and thereby make a content Q-at-issue in the subsequent discourse. This is what is targeted by the `asking whether' diagnostic in \ref{aw} (\citealt{tonhauser_how_2018}), based on the assumption that it is the at-issue content of interrogatives that partitions the context set, as opposed to their non-at-issue content (p.502).

    \ex.[\ref{aw}]%
        \emph{Is Greg, who bought a new car, envied by his neighbor?}\smallskip
    \\ Question to participants: Is the speaker asking whether Greg bought a new car?
    \z.

    explain explain
    If participants respond "no," this suggests that the appositive content (Greg bought a new car) is not part of the at-issue content of the interrogative, providing evidence that it is not Q-at-issue. This diagnostic thus complements the QUD-diagnostic by probing the at-issueness of content from the perspective of explicitly raised questions rather than previously established ones.

    based on the assumption that it is the at-issue content of interrogatives that partitions the context set, as opposed to their non-at-issue content (\citealt{tonhauser_how_2018} p.502).


 
  


\section{Discussion}
  \subsection{Direction}
  \subsection{Speech-act}
  \subsection{Logical (in)dependence between contents}
    \begin{itemize}
      \item Some diagnostics, especially (dis)agreement also interact with speaker commitments

      \item if the embedded content is false participants may choose to disagree with the main assertion, not necessarily because it is interpreted as at-issue, but because it is assumed to be true, and entails that the at-issue content is false.

    \end{itemize}

  \subsection{Koev's dichotomy}
    If we contrast these notions by whether what is at-issue is determined by a declarative vs. an interrogative utterance, the asking whether test aligns more closely with the assumptions of the QUD-based notion of Q-at-issueness. However, Koev suggests that the two notions also differ in their directionality in discourse, arguing that the use of the diagnostics and defitinitions of at-issueness in the literature suggest that Q-at-issueness of a content is determined by the previous discourse (backward-looking), whereas P-at-issueness determines what will be at-issue at the point of the utterance and in subsequent discourse (forward-looking). Considering that the `asking whether' diagnostic assumes that the at-issue content of an interrogative makes a content Q-at-issue in the discourse moving forward, this suggests that the dichotomy between forward-looking P-at-issueness and backward-looking Q-at-issueness might benefit from refining it by considering all logically possible combinations between speech act (assertion vs. question) and directionality in discourse (backward vs. forward).

\nocite{*} %this is to get all the entries of the sample bibliography; delete this line for an actual Glossa submission

%\printbibliography %for use with biblatex; comment out if you use natbib
\bibliography{../at-issueness} %for use with natbib; comment out if you use biblatex, and change 'sample' by the name of your bib-file


\end{document}
