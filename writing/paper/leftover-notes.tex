leftover notes






    \begin{itemize}
      \item \citealt{chen_presuppositions_2024}, Exps. 1,2, QUD diagnostic found that attempting to address questions with the presupposed content of German \enquote{soft triggers} (\emph{entdecken,} ‘discover’; \emph{gewinnen,} ‘win’, \emph{schaffen,} ‘manage to’), whose content can be used to felicitously address a previous question (mean Q+A match rating: $\mu \approx 3.7$ on a five-point scale), whereas \enquote{hard triggers} (\emph{auch,} ‘too’; \emph{wieder,} ‘again’; it-clefts) came with significantly lower ratings ($\approx 2$). This effect was found for adult partipants, but only present tendentially on 4--6 year old children (Exp. 1), and present, but weaker in L2 learners of German with L1 Mandarin Chinese.

      % \item German medial appositive RCs lead to much lower question-answer match ratings ($\mu \approx 2$ ) than main clauses ($\mu \approx 4.1$)\footnote{Notably, their Exp. 1 found that this effect is only present tendentially for 4--6 year old children, and Exp. 2 found that it is present but a bit weaker in adult second language learners of German (with first language Mandarin Chinese).}
      
      \item \citealt{tonhauser_how_2018} found that 

      \emph{stop, only }

      exhibit low at-issueness ratings, although to varying extents.

      their experiment 2 found fine-grained differences between predicates
      

      \item \citealt{xue_correlation_2011}, Exp. 2, use the `yes, but' diagnostics to assess the (non-)at-issueness of the projective content of German expressions \emph{wissen} `know',  \emph{entdecken} `discover',  \emph{auch} `too',  \emph{wieder} `again', when the additive or restitutive presupposition of \emph{auch/wieder} is rejected, there is a roughly $85\%$ preference to still signal agreement with \emph{yes, and} or \emph{yes, but}, suggesting that these inferences are most like construed as non-at-issue content.

      \item \citealt{cummins_backgrounding_2013}, using a version of the `yes, but' diagnostics that uses naturalness ratings found that responses contradicting the presupposition of \emph{too} (and comparative constructions) are more natural with \emph{yes, although} compared to \emph{no, because}, suggesting that the additive presupposition is more often interpreted as not at-issue than as at-issue. In contrast, they found that the presupposition of \emph{again}, as well as those of \emph{continue, regret, still, stop} and \emph{only}, show the reverse preference, although to varying degrees.


    \end{itemize}

    \begin{tabular}{l c c c c}\toprule
            & \emph{too} 
              & \emph{again}
                        \\\midrule
        
        \citealt{chen_presuppositions_2024}, QUD
            & \multirow{2}{*}{NAI}
              & \multirow{2}{*}{NAI}
                        \\ 
        \scriptsize German, 5-point Q+A match rating  &  \\ \midrule

        \citealt{xue_correlation_2011}, `yes, but'
            & \multirow{2}{*}{NAI}
              & \multirow{2}{*}{NAI}
                        \\ 
        \scriptsize German, forced-choice continuation  &  \\ \midrule

        \citealt{cummins_backgrounding_2013}, `yes, but'
            & \multirow{2}{*}{NAI}
              & \multirow{2}{*}{AI}
                        \\ 
        \scriptsize English, 5-point naturalness rating   &  \\ \midrule

    \end{tabular}

  \subsection{clause-embedding predicates}
    \begin{itemize}
      \item \citealt{tonhauser_how_2018}
      \item degen + tonhauser

      \item \citealt{xue_correlation_2011}, Exp. 2, use the `yes, but' diagnostics to assess the (non-)at-issueness of the projective content of German expressions \emph{wissen} `know',  \emph{entdecken} `discover',  \emph{auch} `too',  \emph{wieder} `again',  responses rejecting the complement of \emph{wissen} exhibited a strong preference for using \emph{no} over \emph{yes}, which, according to the authors, suggest that the complement of \emph{wissen} is most often interpreted as at-issue; the complement of\emph{find out}, on the other hand, showed a 50-50 split, suggesting that the predicate introduces no preference about whether its embedded content can be interpreted as at-issue or not-at-issue.

    \end{itemize}

    \begin{itemize}
      \item embedded content in sentences with clause-embedding predicates 
      \item the projection of this content is modulated by its at-issueness
      \item testing the at-issueness of 

    \end{itemize}