leftover notes






    \begin{itemize}
      \item \citealt{chen_presuppositions_2024}, Exps. 1,2, QUD diagnostic found that attempting to address questions with the presupposed content of German \enquote{soft triggers} (\emph{entdecken,} ‘discover’; \emph{gewinnen,} ‘win’, \emph{schaffen,} ‘manage to’), whose content can be used to felicitously address a previous question (mean Q+A match rating: $\mu \approx 3.7$ on a five-point scale), whereas \enquote{hard triggers} (\emph{auch,} ‘too’; \emph{wieder,} ‘again’; it-clefts) came with significantly lower ratings ($\approx 2$). This effect was found for adult partipants, but only present tendentially on 4--6 year old children (Exp. 1), and present, but weaker in L2 learners of German with L1 Mandarin Chinese.

      % \item German medial appositive RCs lead to much lower question-answer match ratings ($\mu \approx 2$ ) than main clauses ($\mu \approx 4.1$)\footnote{Notably, their Exp. 1 found that this effect is only present tendentially for 4--6 year old children, and Exp. 2 found that it is present but a bit weaker in adult second language learners of German (with first language Mandarin Chinese).}
      
      \item \citealt{tonhauser_how_2018} found that 

      \emph{stop, only }

      exhibit low at-issueness ratings, although to varying extents.

      their experiment 2 found fine-grained differences between predicates
      

      \item \citealt{xue_correlation_2011}, Exp. 2, use the `yes, but' diagnostics to assess the (non-)at-issueness of the projective content of German expressions \emph{wissen} `know',  \emph{entdecken} `discover',  \emph{auch} `too',  \emph{wieder} `again', when the additive or restitutive presupposition of \emph{auch/wieder} is rejected, there is a roughly $85\%$ preference to still signal agreement with \emph{yes, and} or \emph{yes, but}, suggesting that these inferences are most like construed as non-at-issue content.

      \item \citealt{cummins_backgrounding_2013}, using a version of the `yes, but' diagnostics that uses naturalness ratings found that responses contradicting the presupposition of \emph{too} (and comparative constructions) are more natural with \emph{yes, although} compared to \emph{no, because}, suggesting that the additive presupposition is more often interpreted as not at-issue than as at-issue. In contrast, they found that the presupposition of \emph{again}, as well as those of \emph{continue, regret, still, stop} and \emph{only}, show the reverse preference, although to varying degrees.


    \end{itemize}

    \begin{tabular}{l c c c c}\toprule
            & \emph{too} 
              & \emph{again}
                        \\\midrule
        
        \citealt{chen_presuppositions_2024}, QUD
            & \multirow{2}{*}{NAI}
              & \multirow{2}{*}{NAI}
                        \\ 
        \scriptsize German, 5-point Q+A match rating  &  \\ \midrule

        \citealt{xue_correlation_2011}, `yes, but'
            & \multirow{2}{*}{NAI}
              & \multirow{2}{*}{NAI}
                        \\ 
        \scriptsize German, forced-choice continuation  &  \\ \midrule

        \citealt{cummins_backgrounding_2013}, `yes, but'
            & \multirow{2}{*}{NAI}
              & \multirow{2}{*}{AI}
                        \\ 
        \scriptsize English, 5-point naturalness rating   &  \\ \midrule

    \end{tabular}

  \subsection{clause-embedding predicates}
    \begin{itemize}
      \item \citealt{tonhauser_how_2018}
      \item degen + tonhauser

      \item \citealt{xue_correlation_2011}, Exp. 2, use the `yes, but' diagnostics to assess the (non-)at-issueness of the projective content of German expressions \emph{wissen} `know',  \emph{entdecken} `discover',  \emph{auch} `too',  \emph{wieder} `again',  responses rejecting the complement of \emph{wissen} exhibited a strong preference for using \emph{no} over \emph{yes}, which, according to the authors, suggest that the complement of \emph{wissen} is most often interpreted as at-issue; the complement of\emph{find out}, on the other hand, showed a 50-50 split, suggesting that the predicate introduces no preference about whether its embedded content can be interpreted as at-issue or not-at-issue.

    \end{itemize}

    \begin{itemize}
      \item embedded content in sentences with clause-embedding predicates 
      \item the projection of this content is modulated by its at-issueness
      \item testing the at-issueness of 

    \end{itemize}


   

\section{Assessing at-issueness}
  \subsection{QUD-diagnostic}
    The QUD diagnostic tests whether a propositional content associated with a declarative assertion (\pref{qud}B) can be interpreted as Q-at-issue by testing whether the assertion is felicitous as an answer to a preceding question that targets that content.
    \begin{itemize}
      \item show here
    \end{itemize}
    % The notion that the at-issueness of a content is related to whether it is used to address the QUD , defined explicitly in \citealt{simons_what_2010}, is labeled Q(uestion)-at-issueness in \citepos{koev_notions_2018} overview:

    \ex. \label{def:qai}%
      Q-at-issueness: \hfill (based on \citealt{simons_what_2010}: 26, \citealt{koev_notions_2018}: 2)\\
        A content $m$ is Q-at-issue in a context $c$ iff
        \a. \label{def:qai-relevant}%
          $m$ is relevant to the QUD in $c$, and
        \b.  \label{def:qai-conventional}%
          $p$ is appropriately conventionally marked relative to the QUD.
        \z. 
      \z.

      Here, $m$ may be either a propositional content or a question meaning. Relevance to the QUD is defined as follows:

      \ex. Relevance to the QUD in context $c$ \hfill (based on \citealt{simons_what_2010}: 13)
        \a. A proposition $p$ is relevant the QUD iff it contextually entails in $c$ a partial or complete answer to the QUD.
        \b. A question $q$ is relevant to the QUD, iff it has an answer that is relevant to the QUD.
        \z.
      \z.

      The QUD-diagnostic from \citealt{tonhauser_diagnosing_2012} operationalizes Q-at-issueness through naturalness judgments. It builds on two assumptions:
      \begin{enumerate}
        \item An overt question explicitly introduces a QUD.\footnote{add reference}
        \item An utterance is felicitous only if its at-issue content is relevant to the QUD (\citealt{amaral_review_2007,tonhauser_diagnosing_2012}).
      \end{enumerate}

      Koev suggests that this diagnostic is \emph{backward-looking}, because it tests whether a given content is at-issue relative to the previous discourse.

      \noindent To test whether a given content $m$ can be construed as Q-at-issue, participants are presented with a context that establishes a QUD via an overt question, followed by a response that includes $m$. For instance, \ref{qud} is used to diagnose the status of the content $m$ of the appositive RC (Greg bought a car) conveyed by B's utterance $U$, by presenting it as a response to a question $Q$ that $m$ is relevant to (What did Greg buy?), and asking a naturalness rating for $U$ as a response to $Q$.

      \ex.[\ref{qud}]
        \a.[A:] \emph{What did Greg buy?}
        \b.[B:] \emph{Greg, who bought a new car, is envied by his neighbor.}
        \z.
        Question to participants: How well does B's response fit A's question?
      \z.

      If $m$ (Greg bought a car) is interpreted as addressing the QUD, the response should receive high naturalness ratings. However, responses like (\pref{qud}B) typically receive low ratings, suggesting that $m$ is not at-issue, that is, even though $m$ is relevant to $Q$ and thereby satisfies the first part of the definition in \ref{def:qai-relevant}. The low naturalness should, therefore, reflect that $m$ is not-at-issue due to the second part of the definition in \ref{def:qai-conventional}: The low ratings for (\pref{qud}B) support the claim that appositive RCs are not appropriately conventionally marked to contribute at-issue content.

      \begin{itemize}
        \item \citealt{chen_presuppositions_2024} used this diagnostic 

        comparing  

      \end{itemize}

  \subsection{Direct dissent and `yes, but' diagnostic}
    Accordingly, the direct-dissent diagnostic \ref{dd} tests whether a proposition associated with the initial utterance (\pref{dd}A) can be interpreted as P-at-issue by testing whether it can be felicitously contradicted using \emph{no}. Relatedly, the `yes, but' diagnostic \ref{yesbut} assesses whether speakers prefer to signal agreement (using \emph{yes}) or disagreement (using \emph{no}) with the main assertion when contradicting the tested content.

    The direct dissent diagnostic \ref{dd} and the `yes, but' diagnostic \ref{yesbut} reflect the notion of P(roposal)-at-issueness, based on the assumption that at-issue content contributes to the main assertion of an utterance, which is taken to constitute a proposal to update the common ground.

     \ex. P-at-issueness: \hfill (\citealt{koev_apposition_2013,koev_notions_2018})\\
      A proposition $p$ is P-at-issue in a context $c$ iff
      \a. $p$ is a proposal in $c$ and
      \b. $p$ has not been accepted or rejected in $c$.
      \z.
    \z.

    Under this conception, the at-issue assertion is the contribution of an utterance that can be directly assented or dissented with using default discourse moves (in the sense of \citealt{farkas_reacting_2010}), for instance, using polar response particles (like English \emph{yes/no}). Conversely, non-at-issue content is assumed to be entailed by the common ground prior to the utterance in question (e.g., presupposed content, \citealt{stalnaker_presuppositions_1973,stalnaker_common_2002}), or imposed on the common ground (\citealt{murray_varieties_2014,anderbois_at-issue_2015}). Importantly, the diagnostics in \ref{dd} and \ref{yesbut} build on the assumption that non-at-issue content requires non-default discourse moves (such as revision, correction, or negotiation) to be dissented with.

    \begin{itemize}
      \item that it should be possible to signal (at least partial) agreement with the main assertion of an utterance even when contradicting is non-at-issue content. It tests whether speakers prefer signaling agreement or disagreement with the previous assertion (using \emph{yes/no}) in repsonses that contradict the content in question.

    \end{itemize}

    \begin{itemize}
      \item These two diagnostics are characterized by \citealt{koev_notions_2018} as \emph{forward-looking}, as they test whether a given content is at-issue relative to utterances in the following discourse.


      \item \citealt{syrett_experimental_2015} Their Exp. 2 found that given a choice to disagree with a preceding main clause or appositive content, participants choose disagreeing with the main clause over a medial appositive around 80\% of the time. However, for final medial RCs, this proportion is reduced to around 65\%. They conclude that final appositive clauses can compete with main clause content in the direct-dissent diagnostic, allowing these contents to be more readily interpreted as at-issue.

      \item \citealt{syrett_experimental_2015}, Exp. 2: used a variant of the direct-dissent diagnostic within a forced-choice continuation task. 
      \item utterance that included some appositive content (illustated here for a medial appositive RC)

      polarity particle \emph{no} to disagree  choice to disagree with the main clause content, or the apposistive content.
    \end{itemize}

    \ex. \a.[A] My friend Sophie, \underline{who performed a piece by Mozart,} is a classical violinist.
      \b.[B1:] No, she’s not. (target: main clause)
      \b.[B2:] No, she didn’t. (target: appositive)
      \z. \z. 

      to avoid concerns that the choice about which proposition to disagree with may be affected by the participants opinion about the content of these propositions, we instead chose a version of the direct dissent task that more directky targets the question whether disagrement using \emph{no} is acceptable, by using acceptabiltiy judgments.

  \subsection{Asking whether}
    This diagnostic tests whether a content associated with a polar question \ref{aw} can be interpreted as at-issue by testing whether informands will understand it as the main issue being asked about.

    Because the definition in \ref{def:qai} references the preceding context, \citet{koev_notions_2018} suggests that QUD-at-issueness is a backward-looking notion of at-issueness. However, overt questions may explicitly raise a QUD\footnote{add reference}, and thereby make a content Q-at-issue in the subsequent discourse. This is what is targeted by the `asking whether' diagnostic in \ref{aw} (\citealt{tonhauser_how_2018}), based on the assumption that it is the at-issue content of interrogatives that partitions the context set, as opposed to their non-at-issue content (p.502).

    \ex.[\ref{aw}]%
        \emph{Is Greg, who bought a new car, envied by his neighbor?}\smallskip
    \\ Question to participants: Is the speaker asking whether Greg bought a new car?
    \z.

    explain explain
    If participants respond "no," this suggests that the appositive content (Greg bought a new car) is not part of the at-issue content of the interrogative, providing evidence that it is not Q-at-issue. This diagnostic thus complements the QUD-diagnostic by probing the at-issueness of content from the perspective of explicitly raised questions rather than previously established ones.

    based on the assumption that it is the at-issue content of interrogatives that partitions the context set, as opposed to their non-at-issue content (\citealt{tonhauser_how_2018} p.502).


    \begin{itemize}

      \item \citealt{destruel_cross-linguistic_2015}: when German medial appositives are contradicted in the following utterance, then most participants choose to signal agreement and contrast \emph{yes, but} ($\approx 90\%$), suggesting NAI status.

      \item \citealt{tonhauser_how_2018}, Medial appositives are among the contents that get the lowest ratings for asking-whether diagnostic in their Exp.1a, suggesting that these are NAI (see also \citealt{solstad_cataphoric_2024})
    \end{itemize}
  


\section{Discussion}
  \subsection{Direction}
  \subsection{Speech-act}
  \subsection{Logical (in)dependence between contents}
    \begin{itemize}
      \item Some diagnostics, especially (dis)agreement also interact with speaker commitments

      \item if the embedded content is false participants may choose to disagree with the main assertion, not necessarily because it is interpreted as at-issue, but because it is assumed to be true, and entails that the at-issue content is false.

    \end{itemize}

  \subsection{Koev's dichotomy}
    If we contrast these notions by whether what is at-issue is determined by a declarative vs. an interrogative utterance, the asking whether test aligns more closely with the assumptions of the QUD-based notion of Q-at-issueness. However, Koev suggests that the two notions also differ in their directionality in discourse, arguing that the use of the diagnostics and defitinitions of at-issueness in the literature suggest that Q-at-issueness of a content is determined by the previous discourse (backward-looking), whereas P-at-issueness determines what will be at-issue at the point of the utterance and in subsequent discourse (forward-looking). Considering that the `asking whether' diagnostic assumes that the at-issue content of an interrogative makes a content Q-at-issue in the discourse moving forward, this suggests that the dichotomy between forward-looking P-at-issueness and backward-looking Q-at-issueness might benefit from refining it by considering all logically possible combinations between speech act (assertion vs. question) and directionality in discourse (backward vs. forward).
