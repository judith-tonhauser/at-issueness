
%!TEX TS-program = xelatex
%!TEX encoding = UTF-8 UnicodeWhat is at-issueness? An experimental comparison of diagnostics

\documentclass[compress, xcolor = dvipsnames, aspectratio=169, handout]{beamer}

%fonts
	\usepackage{fontspec}
	\setmainfont[Scale=MatchLowercase,Mapping=tex-text]{Linux Biolinum}
	\setsansfont[Scale=MatchLowercase,Mapping=tex-text, BoldItalicFeatures={FakeBold=3}]{Linux Biolinum}
	\newfontfamily\opt{Optima}
	\usepackage{pifont}% http://ctan.org/pkg/pifont
	\newcommand{\cmark}{\ding{51}}%
	\newcommand{\xmark}{\ding{55}}%

%metadata
	\title{What is at-issueness?}
	\subtitle{An experimental comparison of diagnostics}
	\author{Conglei Xu, Lisa Hofmann, Judith Tonhauser}
	\institute{\includegraphics[width=.3\textwidth]{../uni-logo-b.png}}
	\date{\small Experiments on the semantics/pragmatics interface (XPrag fest 2025)\\
	    	July 18st, 2025}

%theme
	\usetheme{default}
	\setbeamertemplate{navigation symbols}{}
	\setbeamertemplate{headline}{
		\begin{beamercolorbox}{section in head/foot}
			\vskip2pt\insertnavigation{\paperwidth}\vskip2pt
		\end{beamercolorbox}
	}
	\setbeamertemplate{footline}[frame number]

	\setbeamerfont{title}{family = \opt, series = \bfseries}
	\setbeamerfont{frametitle}{family = \opt, series = \bfseries}
	\setbeamerfont{headline}{family = \opt}

%color scheme
	% screen template
	\definecolor{bgyellow}{HTML}{FFFCF7}
	\definecolor{normaltext}{HTML}{505050}
	\definecolor{highlights}{HTML}{2a6d8c}

	\setbeamercolor{background canvas}{bg=bgyellow}
	\setbeamercolor{normal text}{fg=normaltext}
	\setbeamercolor{palette primary}{fg=highlights}
    \setbeamercolor{palette secondary}{fg=highlights}
    \setbeamercolor{palette tertiary}{fg=highlights}
    \setbeamercolor{palette quaternary}{fg=highlights}
    \setbeamercolor{structure}{fg=highlights}

    \setbeamertemplate{enumerate items}[circle]

% !TeX root = asymmetries-local-contexts.tex
%layout, formatting
	\usepackage[round]{natbib}
	\usepackage{multicol}
	\usepackage{linguex}
	\usepackage{multirow}
	\usepackage{colortbl}
	\usepackage{booktabs}
	\usepackage{array}

%text features
	\usepackage{csquotes}
	\usepackage{stmaryrd}
	\usepackage[normalem]{ulem} % sout

%macros
	\newcommand{\listref}[1]{\phantom{.}\hfill {\scriptsize [#1]}}
	\newcommand{\transl}{\ensuremath\rightsquigarrow}






\begin{document}

\begin{frame}
\titlepage

\end{frame}

\section{Introduction}
	\begin{frame}\frametitle{At-issueness}\small
		Distinguish propositional content expressing the main point of an utterance (at-issue content) from those conveying background information (not-at-issue content)
		\phantom. \hfill {\scriptsize (e.g., \citealt{potts_logic_2005})}

		\ex. \emph{Greg, who bought a new car, is envied by his neighbor.}
			\a. At-issue content: \emph{Greg is envied by his neighbor}
			\b. Not-at-issue content: \emph{Greg bought a new car}\pause
			\z.\z.
		
		\begin{itemize}
			\item Several definitions and diagnostics have been proposed in the literature\\ 
				% \phantom. \hfill
				{\scriptsize (e.g., \citealt{faller_semantics_2003,amaral_review_2007,simons_what_2010,tonhauser_diagnosing_2012,farkas_reacting_2010,murray_varieties_2014,anderbois_at-issue_2015}})
			\pause

			\item 
			\textbf{Do these definitions and diagnostics target the same underlying phenomenon?}\\
			Little discussion in the literature\\
			\phantom. \hfill 
			{\scriptsize e.g., 
			(\citealt{snider_anaphoric_2017,snider_at-issueness_2017,snider_distinguishing_2018,tonhauser_how_2018,koev_notions_2018,faller_discourse_2019,korotkova_evidential_2020})}\pause

			\item This work: take a step in adressing this question, by investigating
			\textbf{whether different commonly used diagnostics yield the same results for propositional contents introduced by the same kinds of expressions?}

			% \item exp 1--4 overview

		\end{itemize}
	
	\end{frame}

	\begin{frame}\frametitle{Preview}
	
		Our results suggest that:
		\begin{itemize}
			\item Exps.~1–4: The diagnostics we test \textbf{do not all yield the same results}
			\item Exps.~5–6: The speech act matters: We get a clearer picture when assessing the not-at issue content of different expressions in \textbf{questions} rather than assertions
		\end{itemize}
	
	\end{frame}

\section{Experiments 1–4}
	\begin{frame}\frametitle{Experiments 1–4}
		\begin{itemize}
			\item Across four experiments, we manipulated the diagnostic used to assess the at-issueness status of the same contents\pause

			\item We chose four diagnostics that have been suggested to target different notions of at-issueness and show different empirical patterns\\ 
			\phantom.\hfill {\scriptsize(\citealt{snider_anaphoric_2017,snider_at-issueness_2017,snider_distinguishing_2018,koev_notions_2018,faller_discourse_2019,korotkova_evidential_2020})}\pause

			\item We offer a systematic experimental comparison

		\end{itemize}
			
	\end{frame}

	\begin{frame}[t]\frametitle{Question-based at-issueness diagnostics}

		Question-based diagnostics assume a conception of at-issueness relative to the QUD: At-issue content is the part of an utterance that interacts with the current QUD\newline 
		\phantom.\hfill {\scriptsize(\citealt{amaral_review_2007,simons_what_2010})}

		% \ex. \label{qud}%
			%     QUD diagnostic (e.g., \citealt{tonhauser_diagnosing_2012,chen_presuppositions_2024})
			%     \a.[A:] \emph{What did Greg buy?}
			%     \b.[B:] \emph{Greg, who bought a new car, is envied by his neighbor.}
			%     \z.
			%     Question to participants: How well does B's response fit A's question?\pause
				% \z.	

			% \ex. \label{aw}%
			%     `asking whether' diagnostic (e.g., \citealt{tonhauser_how_2018,solstad_cataphoric_2024})\smallskip\\
			%       \emph{Is Greg, who bought a new car, envied by his neighbor?}\smallskip
			%   \\ Question to participants: Is the speaker asking whether Greg bought a new car?\pause
			%   \z.

		\begin{center}
			\begin{tabular}{p{.35\linewidth} p{.35\linewidth}}	
				{\visible<1->{\vspace{-.8\baselineskip}\scriptsize Exp.~1 (QUD diagnostic)}} &
				{\visible<2->{\vspace{-.8\baselineskip}\scriptsize Exp.~2 (`asking whether' diagnostic)}}\\
				%
				\visible<1->{\vspace{-.8\baselineskip}\includegraphics[width=\linewidth]{../../writing/paper/figures/trialExp1}}
				&
				\visible<2->{\vspace{-.8\baselineskip}\includegraphics[width=\linewidth]{../../writing/paper/figures/trialExp2}}
				\\
			\end{tabular}
		\end{center}

		\visible<3->{\begin{itemize}
			\item Other diagnostics make other assumptions\newline
			\phantom.\hfill {\scriptsize(\citealt{snider_anaphoric_2017,snider_at-issueness_2017,snider_distinguishing_2018,koev_notions_2018,faller_discourse_2019,korotkova_evidential_2020})}
		\end{itemize}}
	
	\end{frame}

	\begin{frame}[t]\frametitle{Assertion-based at-issueness diagnostics}\small
		Assertion-based diagnostics assume that the at-issue content of an assertion constitues a proposal to update the common ground that can be affirmed or denied\\ 
		\phantom.\hfill {\scriptsize(\citealt{farkas_reacting_2010,murray_varieties_2014,anderbois_at-issue_2015})} \vfill

		  % \ex. \label{dd} Direct dissent diagnostic (e.g., \citealt{tonhauser_diagnosing_2012,syrett_experimental_2015})
		  %   \a.[A:] \emph{Greg, who bought a new car, is envied by his neighbor.}
		  %   \b.[B:]\emph{No, that's not true, he didn't buy a new car.}
		  %   \z.
		  % Question to participants: How natural is B's rejection of A's utterance?\pause
		  % \z.

		  % \ex. \label{yesbut}%
		  %   `yes, but' diagnostic (e.g., \citealt{xue_correlation_2011,destruel_cross-linguistic_2015})
		  %   \a.[A:] \emph{Greg, who bought a new car, is envied by his neighbor.}
		  %   \b.[B:] \emph{Yes, but he didn't buy a new car.} /
		  %   \b.[] \emph{Yes, and he didn't buy a new car.} /
		  %   \b.[] \emph{No, he didn't buy a new car.}
		  %   \z.
		  %   Task for participants: Choose the response that sounds best.
		  % \z.

			\begin{center}
			\begin{tabular}{p{.45\linewidth} p{.45\linewidth}}
				{\visible<1->{\vspace{-.5\baselineskip}\scriptsize Exp.~3 (`direct dissent' diagnostic)}} &
				{\visible<2->{\vspace{-.5\baselineskip}\scriptsize Exp.~4 (`yes, but' diagnostic)}}\\
				%
				\visible<1->{\vspace{-.5\baselineskip}\includegraphics[width=\linewidth]{../../writing/paper/figures/trialExp3}}
				&
				\visible<2->{\vspace{-.5\baselineskip}\includegraphics[width=\linewidth]{../../writing/paper/figures/trialExp4}}
				\\
			\end{tabular}
		\end{center}
	
	\end{frame}

	\begin{frame}[t]\frametitle{Experiments 1–4: contents}\scriptsize
		In seven conditions, each experiment tested propositional contents which previous literature leads us to expect show differences in at-issueness status\medskip\\ \pause

		Propositional contents associated with:\smallskip
		\begin{itemize}
			\item Sentence-medial and sentence-final appositive NRRCs (non-restrictive relative clauses): \citet{syrett_experimental_2015} found that final appositives are more at-issue than medial ones under a version of the direct-dissent diagnostic\smallskip\pause

			\item Clauses embedded by \emph{be right, confirm, discover, confess,} and \emph{know}:
			\citet{degen-tonhauser-glossa} found fine-grained differences in how at-issue these are under the `asking whether' diagnostic\smallskip\pause
		\end{itemize}

		\begin{center}
			\includegraphics[width=0.5\textwidth]{../../results/degen-tonhauser-glossa/graphs/mean-asking-whether-ratings.pdf}
		\end{center}
		
	\end{frame}

	\begin{frame}\frametitle{Experiments 1–4: Method}
		In each experiment, 80 participants saw each of 7 conditions once, each randomly paired with a clause to instantiate it (item), e.g. \emph{Greg bought a new car}, + 2 controls (attention checks)
	
	
	\end{frame}


	\begin{frame}[t]\frametitle{Results}\scriptsize
		\vspace{-.5\baselineskip}
		\begin{minipage}{\textwidth}
		\hspace{-1.18cm}
		\begin{minipage}{\textwidth}
			% \fbox{
			\begin{tabular}{p{.77\textwidth} p{.31\textwidth}}
				\begin{tabular}{p{.37\textwidth} p{.37\textwidth}}
					{\centering Exp.~1 (QUD diagnostic)} &
					{\centering Exp.~2 (`asking whether' diagnostic)} \\
		      		\includegraphics[width=\linewidth]{../../results/exp1/graphs/mean-ratings.pdf}
		      		&
		      		\includegraphics[width=\linewidth]{../../results/exp2/graphs/mean-ratings.pdf}
		      		\\
		      		{\centering Exp.~3 (`direct dissent' diagnostic)} &
					{\centering Exp.~4 (`yes, but' diagnostic)} \\
		      		\includegraphics[width=\linewidth]{../../results/exp3/graphs/mean-ratings.pdf}
		      		&
		      		\includegraphics[width=\linewidth]{../../results/exp4/graphs/mean-ratings.pdf}
		      		\\
				\end{tabular}\pause

				&
			\parbox{\linewidth}{
				\parbox{\linewidth}{\vspace{-2\baselineskip}\scriptsize
				Exp.~2 (`asking whether'): greatest differentiation between contents
				\begin{itemize}
					\item Range of means
					\item More pairwise differences between contents\smallskip
				\end{itemize}

				\uncover<0>{Final NRRCs are not more at issue than medial NRRCs\medskip

				Fine-grained differences only with `asking whether'\medskip

				Content rankings always different
				\begin{itemize}
					% \item Embedded content of \emph{confirm} and \emph{discover} always more at-issue than NRRCs
					\item Most pairwise differences inconsistent
					\item \emph{be right} very different for QUD diagnostic\smallskip
				\end{itemize}
				}
				\uncover<0>{\fbox{\parbox{\linewidth}{\tiny Posthoc pairwise comparisons of the estimated means/proportions for each content using the `emmeans' package (\citealt{emmeans}) in R (\citealt{r}). The input to the pairwise comparisons were bayesian mixed-effects beta regression models (Exps.~1-3) or a mixed-effects logistic regression model (Exp.~4)}}\\ \vfill \phantom. }}
				}
			\end{tabular}
			% }
		\end{minipage}
		\end{minipage}
	
	\end{frame}

	\begin{frame}[t]\frametitle{Results}\scriptsize
		\vspace{-.5\baselineskip}
		\begin{minipage}{\textwidth}
		\hspace{-1.18cm}
		\begin{minipage}{\textwidth}
			% \fbox{
			\begin{tabular}{p{.77\textwidth} p{.31\textwidth}}
				\begin{tabular}{p{.37\textwidth} p{.37\textwidth}}
					{\centering Exp.~1 (QUD diagnostic)} &
					{\centering Exp.~2 (`asking whether' diagnostic)} \\
		      		\includegraphics[width=\linewidth]{../../results/exp1/graphs/mean-ratings.pdf}
		      		&
		      		\includegraphics[width=\linewidth]{../../results/exp2/graphs/mean-ratings.pdf}
		      		\\
		      		{\centering Exp.~3 (`direct dissent' diagnostic)} &
					{\centering Exp.~4 (`yes, but' diagnostic)} \\
		      		\includegraphics[width=\linewidth]{../../results/exp3/graphs/mean-ratings.pdf}
		      		&
		      		\includegraphics[width=\linewidth]{../../results/exp4/graphs/mean-ratings.pdf}
		      		\\
				\end{tabular}

				&
			\parbox{\linewidth}{
				\parbox{\linewidth}{\vspace{-2\baselineskip}\scriptsize
				Exp.~2 (`asking whether'): greatest differentiation between contents
				\begin{itemize}
					\item Range of means
					\item More pairwise differences between contents\smallskip
				\end{itemize}

				\uncover<2->{Final NRRCs are not more at issue than medial NRRCs\medskip}

				\uncover<3->{Fine-grained differences only with `asking whether'\medskip}

				\uncover<4->{Content rankings always different
				\begin{itemize}
					% \item Embedded content of \emph{confirm} and \emph{discover} always more at-issue than NRRCs
					\item Most pairwise differences inconsistent
					\item \emph{be right} very different for QUD diagnostic\smallskip
				\end{itemize}}

				
				}
				\fbox{\parbox{\linewidth}{\tiny Posthoc pairwise comparisons of the estimated means/proportions for each content using the `emmeans' package (\citealt{emmeans}) in R (\citealt{r}). The input to the pairwise comparisons were bayesian mixed-effects beta regression models (Exps.~1-3) or a mixed-effects logistic regression model (Exp.~4)}}\\ \vfill \phantom. 
				}
			\end{tabular}
			% }
		\end{minipage}

		\phantom.
		\hspace{-1.18cm}\begin{minipage}{\textwidth}
			\vspace{-8cm}
			% \fbox{
			\begin{tabular}{p{.77\textwidth} p{.31\textwidth}}
				\begin{tabular}{p{.37\textwidth} p{.35\textwidth}}
					\phantom{\centering Exp.~1 (QUD diagnostic)} &
					\phantom{\centering Exp.~2 (`asking whether' diagnostic)} \\
					%	
						\cellcolor{bgyellow}
						\parbox{\linewidth}{
						\vspace{.2cm}
						\scalebox{.75}{
			      		\begin{tabular}{r | ccccccc}
						    & \rots{be right} & \rots{confirm} & \rots{discover} & \rots{confess} & \rots{know} & \rots{final NRRC} & \rots{medial NRRC} \\
						    \hline
						     be right & \cellcolor{black} & \cellcolor{red} & \cellcolor{white} & \cellcolor{red} & \cellcolor{red} & \cellcolor{red} & \cellcolor{red} \\ 
  confirm & \cellcolor{black} & \cellcolor{black} & \cellcolor{white} & \cellcolor{red} & \cellcolor{white} & \cellcolor{red} & \cellcolor{red} \\ 
  discover & \cellcolor{black} & \cellcolor{black} & \cellcolor{black} & \cellcolor{red} & \cellcolor{white} & \cellcolor{red} & \cellcolor{red} \\ 
  confess & \cellcolor{black} & \cellcolor{black} & \cellcolor{black} & \cellcolor{black} & \cellcolor{blue} & \cellcolor{red} & \cellcolor{white} \\ 
  know & \cellcolor{black} & \cellcolor{black} & \cellcolor{black} & \cellcolor{black} & \cellcolor{black} & \cellcolor{red} & \cellcolor{red} \\ 
  final NRRC & \cellcolor{black} & \cellcolor{black} & \cellcolor{black} & \cellcolor{black} & \cellcolor{black} & \cellcolor{black} & \cellcolor{blue} \\ 
  medial NRRC & \cellcolor{black} & \cellcolor{black} & \cellcolor{black} & \cellcolor{black} & \cellcolor{black} & \cellcolor{black} & \cellcolor{black} \\ 
  

						    \hline
					    \end{tabular}}}
		      		&
		      			\cellcolor{bgyellow}
		      			\parbox{\linewidth}{
						\vspace{.2cm}
						\scalebox{.75}{
			      		\begin{tabular}{r | ccccccc}
						    & \rots{be right} & \rots{confirm} & \rots{discover} & \rots{confess} & \rots{know} & \rots{final NRRC} & \rots{medial NRRC} \\
						    \hline
						     be right & \cellcolor{black} & \cellcolor{red} & \cellcolor{white} & \cellcolor{red} & \cellcolor{red} & \cellcolor{red} & \cellcolor{red} \\ 
  confirm & \cellcolor{black} & \cellcolor{black} & \cellcolor{white} & \cellcolor{red} & \cellcolor{white} & \cellcolor{red} & \cellcolor{red} \\ 
  discover & \cellcolor{black} & \cellcolor{black} & \cellcolor{black} & \cellcolor{red} & \cellcolor{white} & \cellcolor{red} & \cellcolor{red} \\ 
  confess & \cellcolor{black} & \cellcolor{black} & \cellcolor{black} & \cellcolor{black} & \cellcolor{blue} & \cellcolor{red} & \cellcolor{white} \\ 
  know & \cellcolor{black} & \cellcolor{black} & \cellcolor{black} & \cellcolor{black} & \cellcolor{black} & \cellcolor{red} & \cellcolor{red} \\ 
  final NRRC & \cellcolor{black} & \cellcolor{black} & \cellcolor{black} & \cellcolor{black} & \cellcolor{black} & \cellcolor{black} & \cellcolor{blue} \\ 
  medial NRRC & \cellcolor{black} & \cellcolor{black} & \cellcolor{black} & \cellcolor{black} & \cellcolor{black} & \cellcolor{black} & \cellcolor{black} \\ 
  

						    \hline
					    \end{tabular}}
					    \vspace{.1cm}
					    }
		      		\\
		      		\phantom{\centering Exp.~3 (`direct dissent' diagnostic)} &
					\phantom{\centering Exp.~4 (`yes, but' diagnostic)} \vspace{.05cm} \\
		      		%	
						\cellcolor{bgyellow}
						\parbox{\linewidth}{
						\vspace{.2cm}
						\scalebox{.75}{
			      		\begin{tabular}{r | ccccccc}
						    & \rots{be right} & \rots{confirm} & \rots{discover} & \rots{confess} & \rots{know} & \rots{final NRRC} & \rots{medial NRRC} \\
						    \hline
						     be right & \cellcolor{black} & \cellcolor{white} & \cellcolor{white} & \cellcolor{white} & \cellcolor{white} & \cellcolor{blue} & \cellcolor{blue} \\ 
  confirm & \cellcolor{black} & \cellcolor{black} & \cellcolor{white} & \cellcolor{white} & \cellcolor{white} & \cellcolor{blue} & \cellcolor{blue} \\ 
  discover & \cellcolor{black} & \cellcolor{black} & \cellcolor{black} & \cellcolor{white} & \cellcolor{white} & \cellcolor{blue} & \cellcolor{blue} \\ 
  confess & \cellcolor{black} & \cellcolor{black} & \cellcolor{black} & \cellcolor{black} & \cellcolor{white} & \cellcolor{white} & \cellcolor{blue} \\ 
  know & \cellcolor{black} & \cellcolor{black} & \cellcolor{black} & \cellcolor{black} & \cellcolor{black} & \cellcolor{blue} & \cellcolor{blue} \\ 
  final NRRC & \cellcolor{black} & \cellcolor{black} & \cellcolor{black} & \cellcolor{black} & \cellcolor{black} & \cellcolor{black} & \cellcolor{white} \\ 
  

						    \hline
					    \end{tabular}} \vspace{.1cm}}
		      		&
		      		%	
						\cellcolor{bgyellow}
						\parbox{\linewidth}{
						\vspace{.2cm}
						\scalebox{.75}{
			      		\begin{tabular}{r | ccccccc}
						    & \rots{be right} & \rots{confirm} & \rots{discover} & \rots{confess} & \rots{know} & \rots{final NRRC} & \rots{medial NRRC} \\
						    \hline
						     be right & \cellcolor{black} & \cellcolor{red} & \cellcolor{white} & \cellcolor{red} & \cellcolor{red} & \cellcolor{red} & \cellcolor{red} \\ 
  confirm & \cellcolor{black} & \cellcolor{black} & \cellcolor{white} & \cellcolor{red} & \cellcolor{white} & \cellcolor{red} & \cellcolor{red} \\ 
  discover & \cellcolor{black} & \cellcolor{black} & \cellcolor{black} & \cellcolor{red} & \cellcolor{white} & \cellcolor{red} & \cellcolor{red} \\ 
  confess & \cellcolor{black} & \cellcolor{black} & \cellcolor{black} & \cellcolor{black} & \cellcolor{blue} & \cellcolor{red} & \cellcolor{white} \\ 
  know & \cellcolor{black} & \cellcolor{black} & \cellcolor{black} & \cellcolor{black} & \cellcolor{black} & \cellcolor{red} & \cellcolor{red} \\ 
  final NRRC & \cellcolor{black} & \cellcolor{black} & \cellcolor{black} & \cellcolor{black} & \cellcolor{black} & \cellcolor{black} & \cellcolor{blue} \\ 
  medial NRRC & \cellcolor{black} & \cellcolor{black} & \cellcolor{black} & \cellcolor{black} & \cellcolor{black} & \cellcolor{black} & \cellcolor{black} \\ 
  

						    \hline
					    \end{tabular}}\vspace{.1cm}}
		      		\\
				\end{tabular}

				&
				\\ 
			\end{tabular}
			% }
		\end{minipage}
		\end{minipage}
	
	\end{frame}

	\begin{frame}\frametitle{Takeaways and questions}
		\begin{itemize}
			\item We do see differences between diagnostics, importantly not all show the same differences between contents\pause

			\item None of our diagnostics replicate finding from \citet{syrett_experimental_2015} that final appositives are more at issue than medial ones\pause

			\item Only the `asking whether' diagnostic replicates finding from \citealt{degen-tonhauser-glossa} that clause-embedding predicates show fine-grained lexical differences in how at-issue the embedded clause is\pause

		\end{itemize}	\bigskip 

		We further explore this question:

		\begin{center}
			\emph{Why does Exp.~2 `asking whether' show a greater differentiation of contents than the other diagnostics?}
		\end{center}
		
	\end{frame}

	\begin{frame}[t]\frametitle{Why is \enquote{asking whether} so different?}\small 
		Previous literature:
		\begin{itemize}
			\item \textbf{Question-based vs. assertion-based at-issueness} diagnostics:\\
				The question-based diagnostics (QUD, `asking whether') are about whether a proposition is at issue relative to a QUD introduced in the previous discourse, wheras assertion-based diagnostics (direct dissent, `yes but') target a different underlying notion

		\end{itemize}\pause
		\vfill 

		Two other hypotheses: When testing a range of contents, the differentiation between contents (in terms of range of means, significant differences) is greater:
		\begin{enumerate}
			\item \textbf{Question embedding:} \\ 
				\dots when contents are embedded in a question vs. a declarative assertion.\pause
				
			\item \textbf{Response task:} \\ \dots when participants are more directly asked what the utterance was about vs. when giving acceptability judgment or forced-choice continuation responses.
			
		\end{enumerate}

	\end{frame}

\section{Experiments 5–6}

	\begin{frame}[t]\frametitle{Question-embedding or response task?}\scriptsize
	

		In Exps.~5-6, we test whether the fine-grained differences among clause-embedding predicates observed with `asking whether' diagnostic are replicated with other diagnostics.

		\begin{center} \vspace{-.5\baselineskip}
			\begin{tabular}{p{.35\linewidth} p{.35\linewidth}}	
				{\visible<1->{\scriptsize Exp.~5 (`asking whether' diagnostic)}} &
				{\visible<2->{\scriptsize Exp.~6 (`direct response' diagnostic)}} \\
				%
				\visible<1->{\vspace{-.5\baselineskip}\includegraphics[width=\linewidth]{../../writing/paper/figures/trialExp5}}
				&
				\visible<2->{\vspace{-.5\baselineskip}\includegraphics[width=\linewidth]{../../writing/paper/figures/trialExp6}}
				\\
				
			\end{tabular}
		\end{center}
	
		\vfill \pause\pause
		

		\begin{enumerate}
			\item If \textbf{question embedding} is what allows for greater differentiation between contents, results from Exp.~5 and Exp.~6 should be \textbf{\underline{very similar}}\pause

			\item If \textbf{response task} is what allows for greater differentiation between contents, results from Exp.~5 and Exp.~6 should be \textbf{\underline{rather different}}
		\end{enumerate}

	\end{frame}


	\begin{frame}\frametitle{Materials + procedure}

		\begin{itemize}
			\item Contents: 20 clause-embedding predicates of \citet{tonhauser_how_2018} and \citealt{degen-tonhauser-glossa}: \emph{be right, confirm, say, establish, prove, suggest, demonstrate, announce, reveal, admit, confess, think, acknowledge, pretend, see, discover, hear, inform, know} and \emph{be annoyed}\pause

			\item Across the two experiments, 550 participants saw each of 20 conditions once, each randomly paired with a clause to instantiate it (item), e.g. \emph{Julian dances salsa} + 6 control trials (attention checks)


		\end{itemize}

		

	\end{frame}


	% \begin{frame}[t]\frametitle{Results}\scriptsize

	%       \centering
	%       \begin{tabular}{p{.48\linewidth} p{.48\linewidth}}
	%       	Exp.~5 (`asking whether' diagnostic)
	%       	&
	%       	Exp.~6 (`direct response' diagnostic)\\ 
	%       	\includegraphics[width=\linewidth]{../../results/exp5/graphs/mean-ratings.pdf}%
	%       	&
	%       	\includegraphics[width=\linewidth]{../../results/exp6/graphs/mean-ratings.pdf}
	%       	\\
	%       \end{tabular}

	%       % Results of Exps.~5-6. The panels show the mean ratings by expression for (a) Exp.~5 (asking whether diagnostic) and (b) Exp.~2 (direct assent diagnostic). Error bars indicate 95\% bootstrapped confidence intervals. Violin plots show the kernel probability density of individual participants' ratings.
	%       \vfill

	%       \begin{itemize}
	%       	\item We see fine-grained at-issueness distinctions between clause-embedding predicates in both cases
	%       	% \item Pairwise differences bteween many of the same contents (?)\pause
	%       	\item By-content means from the two experiments are highly correlated (Spearman rank correlation = .93)
	%       \end{itemize}
	
	% \end{frame}

	\begin{frame}[t]\frametitle{Results}\scriptsize
		\begin{minipage}{\textwidth}
	      \centering
	      \begin{tabular}{p{.48\linewidth} p{.48\linewidth}}
	      	Exp.~5 (`asking whether' diagnostic)
	      	&
	      	Exp.~6 (`direct response' diagnostic)\\ 
	      	\includegraphics[width=\linewidth]{../../results/exp5/graphs/mean-ratings.pdf}%
	      	&
	      	\includegraphics[width=\linewidth]{../../results/exp6/graphs/mean-ratings.pdf}
	      	\\
	      \end{tabular}
	    \end{minipage}

	    \uncover<2->{
	    \vspace{-5.15cm}
	    \begin{minipage}{\textwidth}
	      \centering
	      \begin{tabular}{p{.48\linewidth} p{.48\linewidth}}
	      	% Exp.~5 (`asking whether' diagnostic)
	      	&
	      	% Exp.~6 (`direct response' diagnostic)
	      	\\ \\ 
	      	\cellcolor{bgyellow}
	      	\scalebox{.6}{
	      	\begin{tabular}{r | ccccc ccccc ccccc ccccc}
			    & \rots{be right} & \rots{confirm} & \rots{say} & \rots{establish} & \rots{prove} & \rots{suggest} & \rots{demonstrate} & \rots{announce} & \rots{reveal} & \rots{admit} & \rots{confess} & \rots{think} & \rots{acknowledge} & \rots{pretend} & \rots{see} & \rots{discover} & \rots{hear} & \rots{inform} & \rots{know} & \rots{be annoyed} \\
			   \hline
			     be right & \cellcolor{black} & \cellcolor{red} & \cellcolor{white} & \cellcolor{red} & \cellcolor{red} & \cellcolor{red} & \cellcolor{red} \\ 
  confirm & \cellcolor{black} & \cellcolor{black} & \cellcolor{white} & \cellcolor{red} & \cellcolor{white} & \cellcolor{red} & \cellcolor{red} \\ 
  discover & \cellcolor{black} & \cellcolor{black} & \cellcolor{black} & \cellcolor{red} & \cellcolor{white} & \cellcolor{red} & \cellcolor{red} \\ 
  confess & \cellcolor{black} & \cellcolor{black} & \cellcolor{black} & \cellcolor{black} & \cellcolor{blue} & \cellcolor{red} & \cellcolor{white} \\ 
  know & \cellcolor{black} & \cellcolor{black} & \cellcolor{black} & \cellcolor{black} & \cellcolor{black} & \cellcolor{red} & \cellcolor{red} \\ 
  final NRRC & \cellcolor{black} & \cellcolor{black} & \cellcolor{black} & \cellcolor{black} & \cellcolor{black} & \cellcolor{black} & \cellcolor{blue} \\ 
  medial NRRC & \cellcolor{black} & \cellcolor{black} & \cellcolor{black} & \cellcolor{black} & \cellcolor{black} & \cellcolor{black} & \cellcolor{black} \\ 
  

			   \hline
		    \end{tabular}}

	      	&
	      	\cellcolor{bgyellow}
	      	\scalebox{.6}{
	      	\begin{tabular}{r | ccccc ccccc ccccc ccccc}
			    & \rots{be right} & \rots{confirm} & \rots{say} & \rots{establish} & \rots{prove} & \rots{suggest} & \rots{demonstrate} & \rots{announce} & \rots{reveal} & \rots{admit} & \rots{confess} & \rots{think} & \rots{acknowledge} & \rots{pretend} & \rots{see} & \rots{discover} & \rots{hear} & \rots{inform} & \rots{know} & \rots{be annoyed} \\
			  \hline
			     be right & \cellcolor{black} & \cellcolor{blue} & \cellcolor{blue} & \cellcolor{blue} & \cellcolor{blue} & \cellcolor{blue} & \cellcolor{blue} & \cellcolor{blue} & \cellcolor{blue} & \cellcolor{blue} & \cellcolor{blue} & \cellcolor{blue} & \cellcolor{blue} & \cellcolor{blue} & \cellcolor{blue} & \cellcolor{blue} & \cellcolor{blue} & \cellcolor{blue} & \cellcolor{blue} & \cellcolor{blue} \\ 
  confirm & \cellcolor{black} & \cellcolor{black} & \cellcolor{white} & \cellcolor{blue} & \cellcolor{blue} & \cellcolor{blue} & \cellcolor{blue} & \cellcolor{blue} & \cellcolor{blue} & \cellcolor{blue} & \cellcolor{blue} & \cellcolor{blue} & \cellcolor{blue} & \cellcolor{blue} & \cellcolor{blue} & \cellcolor{blue} & \cellcolor{blue} & \cellcolor{blue} & \cellcolor{blue} & \cellcolor{blue} \\ 
  say & \cellcolor{black} & \cellcolor{black} & \cellcolor{black} & \cellcolor{blue} & \cellcolor{blue} & \cellcolor{blue} & \cellcolor{blue} & \cellcolor{blue} & \cellcolor{blue} & \cellcolor{blue} & \cellcolor{blue} & \cellcolor{blue} & \cellcolor{blue} & \cellcolor{blue} & \cellcolor{blue} & \cellcolor{blue} & \cellcolor{blue} & \cellcolor{blue} & \cellcolor{blue} & \cellcolor{blue} \\ 
  establish & \cellcolor{black} & \cellcolor{black} & \cellcolor{black} & \cellcolor{black} & \cellcolor{blue} & \cellcolor{blue} & \cellcolor{blue} & \cellcolor{blue} & \cellcolor{white} & \cellcolor{blue} & \cellcolor{blue} & \cellcolor{blue} & \cellcolor{blue} & \cellcolor{blue} & \cellcolor{blue} & \cellcolor{blue} & \cellcolor{blue} & \cellcolor{blue} & \cellcolor{blue} & \cellcolor{blue} \\ 
  prove & \cellcolor{black} & \cellcolor{black} & \cellcolor{black} & \cellcolor{black} & \cellcolor{black} & \cellcolor{white} & \cellcolor{blue} & \cellcolor{white} & \cellcolor{white} & \cellcolor{blue} & \cellcolor{white} & \cellcolor{blue} & \cellcolor{blue} & \cellcolor{blue} & \cellcolor{blue} & \cellcolor{blue} & \cellcolor{blue} & \cellcolor{blue} & \cellcolor{blue} & \cellcolor{blue} \\ 
  suggest & \cellcolor{black} & \cellcolor{black} & \cellcolor{black} & \cellcolor{black} & \cellcolor{black} & \cellcolor{black} & \cellcolor{blue} & \cellcolor{white} & \cellcolor{white} & \cellcolor{blue} & \cellcolor{white} & \cellcolor{blue} & \cellcolor{blue} & \cellcolor{blue} & \cellcolor{blue} & \cellcolor{blue} & \cellcolor{blue} & \cellcolor{blue} & \cellcolor{blue} & \cellcolor{blue} \\ 
  demonstrate & \cellcolor{black} & \cellcolor{black} & \cellcolor{black} & \cellcolor{black} & \cellcolor{black} & \cellcolor{black} & \cellcolor{black} & \cellcolor{red} & \cellcolor{red} & \cellcolor{white} & \cellcolor{red} & \cellcolor{blue} & \cellcolor{blue} & \cellcolor{blue} & \cellcolor{blue} & \cellcolor{blue} & \cellcolor{blue} & \cellcolor{blue} & \cellcolor{blue} & \cellcolor{blue} \\ 
  announce & \cellcolor{black} & \cellcolor{black} & \cellcolor{black} & \cellcolor{black} & \cellcolor{black} & \cellcolor{black} & \cellcolor{black} & \cellcolor{black} & \cellcolor{white} & \cellcolor{blue} & \cellcolor{white} & \cellcolor{blue} & \cellcolor{blue} & \cellcolor{blue} & \cellcolor{blue} & \cellcolor{blue} & \cellcolor{blue} & \cellcolor{blue} & \cellcolor{blue} & \cellcolor{blue} \\ 
  reveal & \cellcolor{black} & \cellcolor{black} & \cellcolor{black} & \cellcolor{black} & \cellcolor{black} & \cellcolor{black} & \cellcolor{black} & \cellcolor{black} & \cellcolor{black} & \cellcolor{blue} & \cellcolor{white} & \cellcolor{blue} & \cellcolor{blue} & \cellcolor{blue} & \cellcolor{blue} & \cellcolor{blue} & \cellcolor{blue} & \cellcolor{blue} & \cellcolor{blue} & \cellcolor{blue} \\ 
  admit & \cellcolor{black} & \cellcolor{black} & \cellcolor{black} & \cellcolor{black} & \cellcolor{black} & \cellcolor{black} & \cellcolor{black} & \cellcolor{black} & \cellcolor{black} & \cellcolor{black} & \cellcolor{white} & \cellcolor{blue} & \cellcolor{blue} & \cellcolor{blue} & \cellcolor{blue} & \cellcolor{blue} & \cellcolor{blue} & \cellcolor{blue} & \cellcolor{blue} & \cellcolor{blue} \\ 
  confess & \cellcolor{black} & \cellcolor{black} & \cellcolor{black} & \cellcolor{black} & \cellcolor{black} & \cellcolor{black} & \cellcolor{black} & \cellcolor{black} & \cellcolor{black} & \cellcolor{black} & \cellcolor{black} & \cellcolor{blue} & \cellcolor{blue} & \cellcolor{blue} & \cellcolor{blue} & \cellcolor{blue} & \cellcolor{blue} & \cellcolor{blue} & \cellcolor{blue} & \cellcolor{blue} \\ 
  think & \cellcolor{black} & \cellcolor{black} & \cellcolor{black} & \cellcolor{black} & \cellcolor{black} & \cellcolor{black} & \cellcolor{black} & \cellcolor{black} & \cellcolor{black} & \cellcolor{black} & \cellcolor{black} & \cellcolor{black} & \cellcolor{white} & \cellcolor{blue} & \cellcolor{blue} & \cellcolor{white} & \cellcolor{white} & \cellcolor{blue} & \cellcolor{blue} & \cellcolor{blue} \\ 
  acknowledge & \cellcolor{black} & \cellcolor{black} & \cellcolor{black} & \cellcolor{black} & \cellcolor{black} & \cellcolor{black} & \cellcolor{black} & \cellcolor{black} & \cellcolor{black} & \cellcolor{black} & \cellcolor{black} & \cellcolor{black} & \cellcolor{black} & \cellcolor{blue} & \cellcolor{blue} & \cellcolor{white} & \cellcolor{white} & \cellcolor{blue} & \cellcolor{blue} & \cellcolor{blue} \\ 
  pretend & \cellcolor{black} & \cellcolor{black} & \cellcolor{black} & \cellcolor{black} & \cellcolor{black} & \cellcolor{black} & \cellcolor{black} & \cellcolor{black} & \cellcolor{black} & \cellcolor{black} & \cellcolor{black} & \cellcolor{black} & \cellcolor{black} & \cellcolor{black} & \cellcolor{white} & \cellcolor{red} & \cellcolor{red} & \cellcolor{white} & \cellcolor{white} & \cellcolor{blue} \\ 
  see & \cellcolor{black} & \cellcolor{black} & \cellcolor{black} & \cellcolor{black} & \cellcolor{black} & \cellcolor{black} & \cellcolor{black} & \cellcolor{black} & \cellcolor{black} & \cellcolor{black} & \cellcolor{black} & \cellcolor{black} & \cellcolor{black} & \cellcolor{black} & \cellcolor{black} & \cellcolor{red} & \cellcolor{white} & \cellcolor{white} & \cellcolor{white} & \cellcolor{blue} \\ 
  discover & \cellcolor{black} & \cellcolor{black} & \cellcolor{black} & \cellcolor{black} & \cellcolor{black} & \cellcolor{black} & \cellcolor{black} & \cellcolor{black} & \cellcolor{black} & \cellcolor{black} & \cellcolor{black} & \cellcolor{black} & \cellcolor{black} & \cellcolor{black} & \cellcolor{black} & \cellcolor{black} & \cellcolor{white} & \cellcolor{blue} & \cellcolor{blue} & \cellcolor{blue} \\ 
  hear & \cellcolor{black} & \cellcolor{black} & \cellcolor{black} & \cellcolor{black} & \cellcolor{black} & \cellcolor{black} & \cellcolor{black} & \cellcolor{black} & \cellcolor{black} & \cellcolor{black} & \cellcolor{black} & \cellcolor{black} & \cellcolor{black} & \cellcolor{black} & \cellcolor{black} & \cellcolor{black} & \cellcolor{black} & \cellcolor{blue} & \cellcolor{blue} & \cellcolor{blue} \\ 
  inform & \cellcolor{black} & \cellcolor{black} & \cellcolor{black} & \cellcolor{black} & \cellcolor{black} & \cellcolor{black} & \cellcolor{black} & \cellcolor{black} & \cellcolor{black} & \cellcolor{black} & \cellcolor{black} & \cellcolor{black} & \cellcolor{black} & \cellcolor{black} & \cellcolor{black} & \cellcolor{black} & \cellcolor{black} & \cellcolor{black} & \cellcolor{white} & \cellcolor{blue} \\ 
  know & \cellcolor{black} & \cellcolor{black} & \cellcolor{black} & \cellcolor{black} & \cellcolor{black} & \cellcolor{black} & \cellcolor{black} & \cellcolor{black} & \cellcolor{black} & \cellcolor{black} & \cellcolor{black} & \cellcolor{black} & \cellcolor{black} & \cellcolor{black} & \cellcolor{black} & \cellcolor{black} & \cellcolor{black} & \cellcolor{black} & \cellcolor{black} & \cellcolor{blue} \\ 
  

			   \hline
			    \end{tabular}}
	      	\\
	      \end{tabular}
	    \end{minipage}}

	      % Results of Exps.~5-6. The panels show the mean ratings by expression for (a) Exp.~5 (asking whether diagnostic) and (b) Exp.~2 (direct assent diagnostic). Error bars indicate 95\% bootstrapped confidence intervals. Violin plots show the kernel probability density of individual participants' ratings.
	      \vfill

	      \begin{itemize}
	      	\item We see fine-grained at-issueness distinctions between clause-embedding predicates in both cases
	      	% \item Pairwise differences bteween many of the same contents (?)\pause
	      	\item By-content means from the two experiments are highly correlated (Spearman rank correlation = .93)
	      \end{itemize}
	
	\end{frame}

	\begin{frame}\frametitle{Discussion}
		
		\begin{itemize}[<+->]
			\item High correlation of results from Exps.~5–6 suggests that the response task (speaker intention judgments vs. acceptability ratings for direct responses) is not what made the `asking whether' diagnostic stand out in Exps.1–4

			\item The \textbf{speech act} (question vs. assertion) affects how sensitive a diagnostic is to differences between contents
			% \begin{itemize}
			% 	\item More than response task
			% 	\item More than question-based vs. assertion-based diagnostics
			% \end{itemize}
			\item Different from previous literature suggesting that diagnostics yield different results:\\
			Highlighted difference between QUD at-issueness, and assertion-based diagnostics, which are either argued not to be about at-issueness at all
			(\citealt{snider_anaphoric_2017,snider_at-issueness_2017,snider_distinguishing_2018}), or about a separate assertion-based notion of at-issueness (\citealt{koev_notions_2018,faller_discourse_2019,korotkova_evidential_2020})
		\end{itemize} 

	\end{frame}	

\section{General discussion}

	\begin{frame}[t]\frametitle{So why are question embeddings so different from assertions?}\scriptsize 
		\begin{itemize}[<+->]
			\item \textbf{Questions:} At-issue content partitions the contect; without speaker commitment\newline \phantom.\hfill {\scriptsize(\citealt{groenendijk_studies_1984,ginzburg_interrogatives_1996,roberts_information_1996,tonhauser_how_2018})}\smallskip

			$\llbracket$\emph{Is Greg, who bought a car, envied by his neighbor?}$\rrbracket =$
			\phantom. \hfill $\{\textnormal{Greg, who bought a car, is envied by his neighbor},$\\
			\phantom. \hfill$\textnormal{Greg, who bought a car, is not envied by his neighbor}\}$\medskip

			\item \textbf{Literal meaning} of appositives / lexical semantics of clause-embedding verbs affects how much/likely the embedded content is interpreted as at issue\smallskip\\ 
			{\scriptsize Big question here: which lexical properties can affect this how? Look to literature on presupposition triggering? (\citealt{abrusan_predicting_2011,schlenker_triggering_2021,anand_facts_2024,scontras_projection_2025})}\medskip

			\item Therefore \textbf{questions are good context to test at-issueness differences} conditioned by literal meaning\medskip
			% (and maybe directive speech acts)

			\item \textbf{Projection:} When propositional content is not-at-issue, it projects, and we understand the speaker to be committed to it\medskip

			\item \textbf{Assertions:} Signal that speaker is committed to the at-issue content, and therefore at-issue content and not-at-issue content are harder to distinguish\medskip

			\item Testing at-issueness of contents in declarative assertions muddles the picture, making at-issue and not-at-issue content harder to distinguish; especially when the test itself invokes commitment (like those using assent/dissent)

		\end{itemize}
	
	\end{frame}
	
	\begin{frame}\frametitle{Takeaways}
	
		\begin{itemize}[<+->]
			\item Experimental confirmation for claims that there are empirical differences between at-issueness diagnostics (\citealt{snider_anaphoric_2017,snider_at-issueness_2017,snider_distinguishing_2018,koev_notions_2018,faller_discourse_2019,korotkova_evidential_2020})
			\begin{center}
				\textbf{It matters which diagnostic is used.}\bigskip
			\end{center}


			\item But: difference between QUD at-issueness and assertion-based diagnostics does not seem to be the most important; \textbf{speech act} where the target content is embedded is more important \medskip

			% \item No replication of \citealt{syrett_experimental_2015}: We did not find that final appositives are more at-issue than medial ones

			\item Important question for future work: What the results might tell us about whether the diagnostics reflect a shared underlying notion of at-issueness
		\end{itemize}

		% \begin{itemize}
			% 	\item these are not relative to a QUD-notion of at-issueness \citealt{snider_anaphoric_2017,snider_at-issueness_2017,snider_distinguishing_2018,koev_notions_2018,faller_discourse_2019,korotkova_evidential_2020}
			% 	\item Snider: not about at-issueness at all
			% 	\item Murray: p-at-issueness
			% \end{itemize}
	
	\end{frame}



\begin{frame}[allowframebreaks]{\bfseries\opt References}
	\footnotesize
	\bibliographystyle{../cslipubs-natbib}
	\bibliography{../../writing/at-issueness}

\end{frame}

\end{document}