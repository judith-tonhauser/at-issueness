% !TEX TS-program = lualatexmk
% glossa-template.tex
% Copyright 2016 Guido Vanden Wyngaerd
%
% This work may be distributed and/or modified under the
% conditions of the LaTeX Project Public License.
% The latest version of this license is in
%   http://www.latex-project.org/lppl.txt
% and version 1.3 or later is part of all distributions of LaTeX
% version 2005/12/01 or later.
%
% This work has the LPPL maintenance status `maintained'.
% 
% The Current Maintainer of this work is 
% Guido Vanden Wyngaerd (guido.vandenwyngaerd@kuleuven.be).
%
% This work consists of the files 
% glossa.cls
% glossa.bst
% gl-authoryear-comp.cbx
% biblatex-gl.bbx
% glossa-template.tex
% glossa.png
%
% The files of the work are derived from the Semantics & Pragmatics style files
% by Kai von Fintel, Christopher Potts, and Chung-chieh Shan
% All changes are documented on the github repository 
% https://github.com/guidovw/Glossalatex.

\PassOptionsToPackage{table}{xcolor}
\documentclass[times,linguex,xcolor]{glossa}
\usepackage{rotating}
\usepackage{tablefootnote}
\usepackage{colortbl}
\usepackage{color}
\usepackage{multicol}
\usepackage{booktabs}

\usepackage{adjustbox}
\usepackage{array}

\newcolumntype{R}[2]{%
    >{\adjustbox{angle=#1,lap=\width-(#2)}\bgroup}%
    l%
    <{\egroup}%
}

\newcommand*\rots{\multicolumn{1}{R{90}{.7em}}}% no optional argument here, please!
%\usepackage{xcolor}

% possible options:
% [times] for Times font (default if no option is chosen)
% [cm] for Computer Modern font
% [lucida] for Lucida font (not freely available)
% [brill] open type font, freely downloadable for non-commercial use from http://www.brill.com/about/brill-fonts; requires xetex
% [charis] for CharisSIL font, freely downloadable from http://software.sil.org/charis/
% for the Brill an CharisSIL fonts, you have to use the XeLatex typesetting engine (not pdfLatex)
% for headings, tables, captions, etc., Fira Sans is used: https://www.fontsquirrel.com/fonts/fira-sans
% [biblatex] for using biblatex (the default is natbib, do not load the natbib package in this file, it is loaded automatically via the document class glossa.cls)
% [linguex] loads the linguex example package
% !! a note on the use of linguex: in glossed examples, the third line of the example (the translation) needs to be prefixed with \glt. This is to allow a first line with the name of the language and the source of the example. See example (2) in the text for an illustration.
% !! a note on the use of bibtex: for PhD dissertations to typeset correctly in the references list, the Address field needs to contain the city (for US cities in the format "Santa Cruz, CA")

%\addbibresource{sample.bib}
% the above line is for use with biblatex
% replace this by the name of your bib-file (extension .bib is required)
% comment out if you use natbib/bibtex

\let\B\relax %to resolve a conflict in the definition of these commands between xyling and xunicode (the latter called by fontspec, called by charis)
\let\T\relax
\usepackage{xyling} %for trees; the use of xyling with the CharisSIL font produces poor results in the branches. This problem does not arise with the packages qtree or forest.
%\usepackage[linguistics]{forest} %for nice trees!


% \pdf* commands provide metadata for the PDF output. ASCII characters only!
\pdfauthor{}
\pdftitle{What is at-issueness?}
\pdfkeywords{}

\title[What is at-issueness?]{What is at-issueness? An experimental comparison of diagnostics\\ 
  % \bigskip \large Word count: 4720
  }
% Optional short title inside square brackets, for the running headers.

\author[]% short form of the author names for the running header. If no short author is given, no authors print in the headers.
{%as many authors as you like, each separated by \AND.
  % \spauthor{Waltraud Paul\\
  % \institute{CRLAO, CNRS-EHESS-INALCO}\\
  % \small{%105, Bd. Raspail, 75005 Paris\\
  % waltraud.paul@ehess.fr}
  % }
  % \AND
  % \spauthor{Guido Vanden Wyngaerd \\
  % \institute{KU Leuven}\\
  % \small{%Warmoesberg 26, 1000 Brussel\\
  % guido.vandenwyngaerd@kuleuven.be}
  % }%
}


%=====================================================================
%=========================== text ===========================

	% punctuation
		\usepackage{csquotes} % for quotation marks

%====================================================================
%=========================== links, references =======================
	% more linguex options for referencing select examples without parentheses
	  \newif\ifparens\parensfalse
	  \makeatletter
	  \renewcommand{\theExNo}{\protect\theExLBr\arabic{ExNo}\protect\theExRBr}
	  \renewcommand{\theSubExNo}{%
	    \hbox{\if@noftnote\protect\theExLBr\Exarabic{ExNo}\firstrefdash
	        \Exalph{SubExNo}\protect\theExRBr
	      \else
	        \protect\theFnExLBr\Exroman{FnExNo}\firstrefdash%
	        \Exalph{SubExNo}\protect\theFnExRBr
	      \fi}}

	  \renewcommand{\theSubSubExNo}{%
	    \hbox{\if@noftnote\protect\theExLBr%
	            \Exarabic{ExNo}\firstrefdash\Exalph{SubExNo}\secondrefdash
	               \Exroman{SubSubExNo}\protect\theExRBr%
	      \else\protect\theFnExLBr\Exroman{FnExNo}\firstrefdash
	                \Exalph{SubExNo}\secondrefdash\Exarabic{SubSubExNo}\protect\theFnExRBr\fi}}%
	  \makeatother
	  \renewcommand\theExLBr{\ifparens\else(\fi}
	  \renewcommand\theExRBr{\ifparens\else)\fi}
	  \newcommand\pref[1]{{\parenstrue\ref{#1}}}

	% in text citation macros
	\newcommand{\citepos}[1]{\citeauthor{#1}'s \citeyear{#1}}
	\newcommand{\citeposs}[1]{\citeauthor{#1}'s}
	\newcommand{\citetpos}[1]{\citeauthor{#1}'s \citeyear{#1}}

	%
	\usepackage{cleveref}



%=====================================================================
%=========================== figures, tables =========================

	\usepackage{enumitem}


%=====================================================================
%=========================== figures, tables =========================
	% \usepackage{subcaption}
	\usepackage{multirow}

% positive coefficients/difference
\definecolor{red}{RGB}{178,24,43}

% negative coefficients/difference
\definecolor{blue}{RGB}{33,102,172}

% comments by JT
\newcommand{\jt}[1]{\textbf{\color{orange}JT: #1}}

\begin{document}


\maketitle


\begin{abstract}
  at-issueness is a key concept in theoretical semantics/pragmatics, but there is no consensus about how it is defined or diagnosed (e.g., \citealt{tonhauser_diagnosing_2012,tonhauser_how_2018,koev_notions_2018}). We present experimental data investigating whether four widely used diagnostics for at-issueness yield consistent results. Our findings reveal significant differences across diagnostics, indicating they are not interchangeable. Since the diagnostics target distinct theoretical conceptions of at-issueness, these differences offer insight into their comparability.

\end{abstract}

% \begin{keywords}
%   at-issueness, experimental pragmatics, discourse interpretation
% \end{keywords}

\section{Introduction \label{sec:1_introduction}}


  

  

  

  % These are pressing questions for research on at-issueness (see, e.g., \citealt{tonhauser_how_2018,koev_notions_2018,faller_discourse_2019,korotkova_evidential_2020}).

  %
  Using the direct-dissent diagnostic, medial appositives consistently behave as not-at-issue across multiple languages and methods, including fieldwork elicitation for Paraguayan Guaraní (\citealt{tonhauser_diagnosing_2012}), a forced-choice continuation task in English (\citealt{syrett_experimental_2015}), and impressionistic judgments in English (\citealt{potts_logic_2005,amaral_review_2007}). The same conclusion emerges for German medial appositives with the `yes, but' diagnostic in a forced-choice continuation task (\citealt{destruel_cross-linguistic_2015}), and the `asking whether' diagnostic concurs by classifying English medial appositives as clearly not-at-issue (\citealt{tonhauser_how_2018,solstad_cataphoric_2024}).\jt{this reads as if at-issueness is categorical. i think we need to be a bit more nuanced.}

  For instance, medial appositives, which are commonly characterized as not-at-issue have been found to pattern as such when tested with the direct-dissent diagnostic (\citealt{syrett_experimental_2015}), but they have also been argued to sometimes pattern as at-issue when tested with the QUD-diagnostic (\citealt{koev_notions_2018}).

  Final appositives, in contrast, have been characterized as not-at-issue when tested with the QUD-diagnostic (\citealt{koev_notions_2018}), but \citealt{syrett_experimental_2015} results suggest that medial ones are less at-issue than final ones based on the direct dissent.


  \begin{itemize}

  \item On the diagnostics-side, Tonhauser et al 2018: comparison of asking-whether and are you sure diagnostic suggests some differences (p.526ff)

  \item On the diagnostics-side, Tonhauser et al 2018 asking whether didn't find difference in exp 1a between know and discover (see also Solstad/Bott) but Xue/Onea found difference between wissen and entdecken

  \item On the diagnostics-side, Degen/Tonhauser 2025 found fine-grained differences between CCs of clause-embedding predicates using the asking-whether diagnostic, where Hooper 1975 distinguished weak assertive (e.g., think, acknowledge), strong assertive (e.g., say, suggest), semi-factive (assertive; e.g., see, discover, know) and factive (nonassertive; e.g., be annoyed).

  \end{itemize}

  On the theoretical side, the QUD-based view and assertion-based view of at-issueness have been compared and contrasts

  \begin{itemize}
    \item due to finding empirical differences between the diagnostics have been interpreted as suggesting that they the two kinds of diagnostics do not target the same underlying notion (e.g., \citealt{snider_anaphoric_2017,snider_distinguishing_2018,koev_notions_2018,faller_discourse_2019,korotkova_evidential_2020})

    \item \citealt{tonhauser_how_2018}, while finding some differences, also find commonalities, showing that the at-issueness measures obtained from the question-based asking-whether diagnostic, and an additional assertion-based diagnostic used there (\emph{Are you sure?} test)both correlate with projection measures (assessing speaker commitment to embedded contents), and to some extent correlate with each other, which is there taken as suggesting that they are sensitive to a shared underlying phenomenon

    \item we can see that the theoretical and empirical questions have often been addressed together, although there are also interpretations of data which show distinct ratings (for medial/final) appositives that are compatible with the same underlying notion of at-issueness, but different ways they interact with the surrounding discourse (jasinskaya not at issue any more); while anaphoric potential does not amount to at-issueness, there are ways to test the agreeingability / challengability of an assertion without anaphora

  \end{itemize}


\begin{itemize}[leftmargin=12pt]



\item To date, there has not been a systematic comparison of diagnostics to understand whether different diagnostics yield the same results.

\item This paper takes a first step towards addressing the important questions pointed out above. Specifically, we compare the results of diagnostics to identify whether diagnostics yield the same results. As these diagnostics also differ in their underlying assumptions about at-issueness, the results of our experiments also bear on the question of whether currently available definitions define the same underlying concept.

\item The way we do this is by investigating the at-issueness of the same contents with several different diagnostics. 

\begin{itemize}

\item Exps 1-4:  

\begin{itemize}

\item Contents: Sentence-medial and -final NRRCs because one diagnostic found a difference, CCs of clause-embedding predicates because another diagnostic found fine-grained differences

\item Are these differences between contents replicated by the use of other diagnostics?

\end{itemize}

\item Exps 5-6:

\begin{itemize}

\item Contents: CCs of clause-embedding predicates because one diagnostic found fine-grained differences

\item Are these differences between contents replicated by the use of other diagnostics?

\end{itemize}

\end{itemize}

\item A note on terminology before we get started:

\begin{itemize}

\item The theoretical concept of at-issueness may be a categorical notion or a gradient one (see, e.g., Tonhauser et al 2018, Ebert/xx 2024), we remain agnostic here.

\begin{itemize}

\item Categorical: e.g., content is at-issue iff it addresses the QUD, not-at-issue otherwise; gradient mean ratings could be attributed to uncertainty about what the QUD is

\item Gradient: e.g., the extent to which content is at-issue is the extent to which it is relevant to the QUD; gradient mean ratings may be taken to reflect gradient relevance.

\end{itemize}

\item We will continue talking about the diagnostics as diagnostics for at-issueness, even if they may ultimately turn out to not be diagnosing the same underlying concept.

\item For each diagnostic, we will collect ratings: e.g., under the `asking-whether' diagnostic we collect asking-whether ratings, under the QUD diagnostic we collect naturalness ratings. We compare the mean ratings of the contents under each diagnostic; given that we are aggregating over multiple items and participants (and ratings on a scale for some), we may say that the mean rating for one content is higher/lower than that of another, or that they do not differ. 

\item We will take these results to suggest that, given a particular diagnostic, one content is more/less at-issue than another or that the two contents don't differ in at-issueness. We will do this even though the diagnostics may ultimately diagnose different concepts. Also, this does not commit us to a categorical or gradient theoretical notion of at-issueness (see above).

\end{itemize}

\end{itemize} 

%\printbibliography %for use with biblatex; comment out if you use natbib
\bibliography{../at-issueness} %for use with natbib; comment out if you use biblatex, and change 'sample' by the name of your bib-file
\end{document}
%%% Local Variables:
%%% mode: latex
%%% TeX-master: t
%%% TeX-engine: luatex
%%% End: