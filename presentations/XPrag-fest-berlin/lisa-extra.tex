% !TeX root = asymmetries-local-contexts.tex
%layout, formatting
	\usepackage[round]{natbib}
	\usepackage{multicol}
	\usepackage{linguex}
	\usepackage{multirow}
	\usepackage{colortbl}
	\usepackage{booktabs}
	\usepackage{array}
	\usepackage{hyperref}

%text features
	\usepackage{csquotes}
	\usepackage{stmaryrd}
	\usepackage[normalem]{ulem} % sout

%macros
	\newcommand{\listref}[1]{\phantom{.}\hfill {\scriptsize [#1]}}
	\newcommand{\transl}{\ensuremath\rightsquigarrow}

% drawing 
	\usepackage{tikz}
	\usepackage{venndiagram}
	\usepackage{easybmat}

% diagram shortcuts
	\usetikzlibrary{calc}
	\tikzstyle{index gray}=[inner sep=2pt, black, circle, fill=lightgray]
	\tikzstyle{opaque}=[fill=gray,fill opacity=.1]
	\newcommand{\indices}{% Indices
	    \draw (-1,1) node[index gray] (yy) {$w_{us}$};
	    \draw (1,1) node[index gray] (yn) {$w_u\phantom s$};
	    \draw (-1,-1) node[index gray] (ny) {$w_s\phantom u$};
	    \draw (1,-1) node[index gray] (nn) {$w_0\phantom i$};
		}


% propositions
	\newcommand{\sprop}{% s proposition
		\draw[rounded corners] (-1.9,1.9) rectangle (-.1, -1.9);}
	\newcommand{\uprop}{% u proposition
		\draw[rounded corners] (-1.9,1.9) rectangle (1.9,.1);}
	\newcommand{\nsprop}{% ~s proposition
		\draw[rounded corners] (1.9,-1.9) rectangle (.1, 1.9);}

% contexts
	\newcommand{\ccontext}{% c context
		\draw[opaque, rounded corners] (-1.8,1.8) rectangle (1.8, -1.8);}

	\newcommand{\ucontext}{% u context
		\draw[opaque, rounded corners] (-1.8,1.8) rectangle (1.8,.2);}
	\newcommand{\nucontext}{% ~u context
		\draw[opaque, rounded corners] (-1.8,-1.8) rectangle (1.8,-.2);}
	\newcommand{\scontext}{% s context
		\draw[opaque, rounded corners] (-1.8,1.8) rectangle (-.2, -1.8);}
	\newcommand{\nscontext}{% ~s context
		\draw[opaque, rounded corners] (1.8,1.8) rectangle (.2, -1.8);}
	
	\newcommand{\unscontext}{% u + ~s context
		\draw[opaque, rounded corners] (.2,1.8) rectangle (1.8,.2);}
	\newcommand{\uscontext}{% u + s context
		\draw[opaque, rounded corners] (-.2,1.8) rectangle (-1.8,.2);}
	\newcommand{\nuscontext}{% ~u + s context
		\draw[opaque, rounded corners] (-.2,-1.8) rectangle (-1.8,-.2);}
	\newcommand{\zerocontext}{% u + ~s context
		\draw[opaque, rounded corners] (.2,-1.8) rectangle (1.8,-.2);}

	\newcommand{\undefined}{% undefined context
		\draw[thick, Magenta] (-1.9,1.9) -- (1.9,-1.9);
		\draw[thick, Magenta] (-1.9,-1.9) -- (1.9,1.9);}
	
	\newcommand{\negquitwoqud}{% c context
		\draw[opaque, rounded corners] (-1.8,1) -- (-1.8,1.8) -- (.1,1.8) -- (.1,.1) -- (1.8,.1) -- (1.8, -1.8) -- (-.1, -1.8) -- (-.1, -.1) -- (-1.8, -.1) -- (-1.8,1);}
	\newcommand{\nquitcontext}{% not quit context
		\draw[opaque, rounded corners] (-1.8,0) -- (-1.8,1.8) -- (-.2,1.8) -- (-.2,-.2) -- (.2,-.2) -- (1.8,-.2) -- (1.8, -1.8) -- (-1.8, -1.8) -- 
		(-1.8,0);}

% local contexts
	\newcommand{\uslocal}{% u+s local context
		\draw[rounded corners] (-1.8,1.8) rectangle (-.2,.2);}
	\newcommand{\unslocal}{% u+~ns local context
		\draw[rounded corners] (1.8,1.8) rectangle (.2,.2);}
	\newcommand{\unslocalnudge}{% u+~ns local context
		\draw[rounded corners, xshift=-.1cm,yshift=-.1cm ] (1.8,1.8) rectangle (.2,.2);}
	

% input contexts
	\newcommand{\cinput}{% c input context
		\draw[dashed, rounded corners] (-1.9,1.9) rectangle (1.9, -1.9);}
	\newcommand{\uinput}{% u input context
		\draw[dashed, rounded corners] (-1.9,1.9) rectangle (1.9,.1);}
	\newcommand{\nuinput}{% u input context
		\draw[dashed, rounded corners] (-1.9,-1.9) rectangle (1.9,-.1);}
	\newcommand{\sinput}{% s input context
		\draw[dashed, rounded corners] (-1.9,1.9) rectangle (-.1, -1.9);}
	\newcommand{\nsinput}{% ~s input context
		\draw[dashed, rounded corners] (1.9,-1.9) rectangle (.1, 1.9);}


	% distance graph
	\newcommand{\distancegraph}{
    	\draw[->] (yn) -- (yy);   % yy → yn
    	\draw[->] (yn) -- (nn);   % yn → nn
	    \draw[->] (yy) -- (ny);   % yy → ny
	    \draw[->] (nn) -- (ny);   % ny → nn
        \draw (yn) circle (0.33);         % inner circle (node border is ~0.25 cm)
	    \draw (yn) circle (0.39);
		}
	\newcommand{\classes}{	
		\draw[shorten <= -1cm, shorten >= -1cm] ($(yy)!0.5!(yn)$) -- ($(yn)!0.5!(nn)$);
		\draw[shorten <= -1cm, shorten >= -1cm] ($(yy)!0.5!(ny)$) -- ($(ny)!0.5!(nn)$);}
	\newcommand{\classlabels}{
		\node at ($(yn)+(0.8,0.5)$) {$d=0$};                 % region containing w_u
		\node at ($(yy)!0.5!(nn)$)   {$d=1$};                 % region containing w_{us}, w_0
		\node at ($(ny)+(-0.8,-0.5)$){$d=2$}; 
	}


